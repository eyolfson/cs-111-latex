\documentclass{article}

\usepackage{enumitem}
\usepackage{fontspec}
\usepackage[letterpaper,margin=72pt]{geometry}
\usepackage{hyperref}
\usepackage{import}
\usepackage{listings}
\usepackage{xcolor}

\subimport{../}{colors.tex}

\setsansfont{Overpass}[Scale=MatchLowercase]
\setmonofont{Overpass Mono}[Scale=MatchLowercase]

\renewcommand{\familydefault}{\sfdefault}

\hypersetup{
  colorlinks=true,
  urlcolor=uclablue,
}

\setlist{nosep}

\makeatletter
\newcommand\version[1]{\renewcommand\@version{#1}}
\newcommand\@version{}

\newcommand\labnumber[1]{\renewcommand\@labnumber{#1}}
\newcommand\@labnumber{}

\newcommand\duedate[1]{\renewcommand\@duedate{#1}}
\newcommand\@duedate{}

\renewcommand\maketitle{%
  \noindent
  {\Large \color{uclablue} CS 111: Operating System Principles}

  \noindent
  {\Large \color{uclablue} Lab \@labnumber}\\[-0.75em]

  \noindent
  {\Huge \bfseries \color{uclablue} \@title}
  {\ttfamily \footnotesize \color{uclablue} \@version}\\[-0.75em]

  \noindent
  {\@author}

  \noindent
  {\@date}

  \noindent
  {Due: \@duedate}\\[1em]
}
\makeatother

\lstset{
  basicstyle=\ttfamily,
}


\lecturenumber{12}
\title{Midterm Review}
\version{1.0.0}
\author{Jon Eyolfson}
\date{April 27, 2021}

\begin{document}
  \begin{frame}[plain, noframenumbering]
    \titlepage
  \end{frame}

  \begin{frame}
    \frametitle{The Midterm Will Be Open for 24 Hours}

    Anytime on the April 29 in PST

    \hspace{2em} (12:00 AM  to 11:59 PM)

    \vspace{2em}

    The midterm will be 2 hours from when you start

    \hspace{2em} No pausing and coming back

    \vspace{2em}

    It'll be open book

    \hspace{2em} I'm assuming there's no way to stop that
  \end{frame}

  \begin{frame}
    \frametitle{The Exam Will Have 3 Types of Questions}

    There will be:
    \begin{itemize}
      \item 15 fill in the term / multiple choice questions
      \item 6 short answers
      \item 2 long answers
    \end{itemize}

    \vspace{2em}

    It'll cover everything up and including Lecture 11

    \vspace{2em}

    It'll be graded out of 100
  \end{frame}

  \begin{frame}
    \frametitle{Winter 2020 Midterm is Good Indicator}

    We'll go over that, except for question 6 and 7

    \vspace{1em}

    For question 6, it's the clock algorithm we went over

    \hspace{2em} It was basically FIFO

    \vspace{1em}

    For question 7, we didn't talk about emulation

    \hspace{1em} Maybe we'll talk about Windows Subsystem for Linux later in the
    course
  \end{frame}

  \begin{frame}
    \frametitle{Question 1}

    What is the benefit of using the copy-on-right optimization when performing
    a fork in the Linux system? 
  \end{frame}

  \begin{frame}
    \frametitle{Question 2}

    Round Robin, First come First Serve, and Shortest Job First are three
    scheduling algorithms that can be used to schedule a CPU.

    \vspace{2em}

    What are their advantages and disadvantages? Which one is likely to have the
    largest overhead? Why?
  \end{frame}

  \begin{frame}
    \frametitle{Question 3 (1)}

    Assume you have a system with three processes (X, Y, and Z) and a single
    CPU.

    Process X has the highest priority, process Z has the lowest, and Y is in
    the middle.

    \vspace{2em}

    Assume a priority-based scheduler (i.e., the scheduler runs the highest
    priority job, performing preemption as necessary).
    Processes can be in one of five states: RUNNING, READY, BLOCKED, not yet
    created, or terminated.

    \vspace{2em}

    Given the following cumulative timeline of process behavior, indicate the
    state the specified process is in AFTER that step, and all preceding steps,
    have taken place. Assume the scheduler has reacted to the specified
    workload change.
  \end{frame}

  \begin{frame}
    \frametitle{Question 3 (2)}

    For all questions in this Part, use the following options for each answer:
    \begin{itemize}
      \item RUNNING
      \item READY
      \item BLOCKED
      \item Process has not been created yet
      \item Not enough information to determine
      \item None of the above
    \end{itemize}
  \end{frame}

  \begin{frame}
    \frametitle{Question 3 (3)}

    \begin{itemize}
      \item Process X is loaded into memory and begins; it is the only
            user-level process in the system. Process X is in which state?
      \item Process X calls fork() and creates Process Y.Process X is in which
            state?
      \item The running process issues an I/O request to the disk. Process X is
            in which state? Process Y is in which state?
      \item The running process calls fork() and creates process Z. Process X is
            in which state? Process Y is in which state? Process Z is in which
            state?
      \item The previously issued I/O request completes. Process X is in which
            state? Process Y is in which state? Process Z is in which state?
    \end{itemize}
  \end{frame}

  \begin{frame}[fragile]
    \frametitle{Question 4}

    For the next two questions, assume the following code is compiled and run on
    a modern linux machine (assume any irrelevant details have been omitted):

    \begin{lstlisting}
main() {
  int a = 0;
  int rc = fork();
  a++;
  if (rc == 0) { rc = fork(); a++; }
  else { a++; }
  printf(“Hello!\n”);
  printf(“a is %d\n”, a);
}
    \end{lstlisting}

    \begin{itemize}
      \item Assuming fork() never fails, how many times will the message
            \lstinline|Hello!\n| be displayed? Explain how you get this result.
      \item What will be the largest value of “a” displayed by the program?
            Explain how you get this result.
    \end{itemize}
  \end{frame}

  \begin{frame}
    \frametitle{Question 5}

    Consider the following set of processes, with associated processing times
    and priorities:

    \begin{center}
      \footnotesize
      \begin{tabular}{lrr}
        Process & Burst Time & Priority \\
        A & 4 & 3 \\
        B & 1 & 1 \\
        C & 2 & 3 \\
        D & 1 & 4 \\
        E & 4 & 2 \\
      \end{tabular}
    \end{center}

    All of the processes arrive at time 0 in the order Process A, B, C, D, E.

    \vspace{1em}

    For each scheduling algorithm, fill in the table with the process that is
    running on the CPU (for time slice-based algorithms, assume a 1 unit time
    slice).

    \vspace{1em}

    \textit{Table says to do: FIFO, Round Robin, SRTF, Priority}
  \end{frame}

  \begin{frame}
    \frametitle{We'll Quickly Run Over Lectures 1 to 11}

    Most of the concepts are important, you'll have to know the terminology

    \vspace{2em}

    Don't rely too much on the exam being open book

    \hspace{2em} If you're looking up everything you'll run out of time

    \vspace{2em}

    You'll want to understand what you did in Lab 0 and 1
  \end{frame}
\end{document}
