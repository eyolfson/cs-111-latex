\documentclass[aspectratio=169]{beamer}

\usepackage{ccicons}
\usepackage{fontspec}
\usepackage{import}
\usepackage{listings}
\usepackage{tikz}

\subimport{../}{colors.tex}

\usetikzlibrary{
  arrows,
  arrows.meta,
  automata,
  backgrounds,
  calc,
  chains,
  decorations.pathreplacing,
  fit,
  matrix,
  overlay-beamer-styles,
  positioning,
  shapes,
  tikzmark,
}
\usetikzmarklibrary{listings}

\hypersetup{
  colorlinks=true,
  urlcolor=uclablue,
}

\setbeamercolor{frametitle}{fg=primarycolor}
\setbeamercolor{structure}{fg=primarycolor}
\setbeamercolor{enumerate item}{fg=black}
\setbeamercolor{itemize item}{fg=black}
\setbeamercolor{itemize subitem}{fg=black}

\setbeamersize{text margin left=26.6mm}
\addtolength{\headsep}{2mm}

\setbeamertemplate{navigation symbols}{}
\setbeamertemplate{headline}{}
\setbeamertemplate{footline}{}
\setbeamertemplate{itemize item}{\color{black}}
\setbeamertemplate{itemize items}[circle]

\setbeamertemplate{footline}{
  \begin{tikzpicture}[remember picture,
                      overlay,
                      shift={(current page.south west)}]
    \node [black!50, inner sep=2mm, anchor=south east]
          at (current page.south east) {\footnotesize \insertframenumber};
  \end{tikzpicture}
}

\setsansfont{Overpass}[Scale=MatchLowercase]
\setmonofont{Overpass Mono}[Scale=MatchLowercase]

\makeatletter
\newcommand\version[1]{\renewcommand\@version{#1}}
\newcommand\@version{}
\def\insertversion{\@version}

\newcommand\lecturenumber[1]{\renewcommand\@lecturenumber{#1}}
\newcommand\@lecturenumber{}
\def\insertlecturenumber{\@lecturenumber}
\makeatother

\setbeamertemplate{title page}
{
  \begin{tikzpicture}[remember picture,
                      overlay,
                      shift={(current page.south west)},
                      background rectangle/.style={fill=uclablue},
                      show background rectangle]
    \node [anchor=west, align=left, inner sep=0, text=white]
          (lecturenumber) at (\paperwidth / 6, \paperheight * 3 / 4)
          {\Large Lecture \insertlecturenumber};
    \node [inner sep=0, align=left, text=white, node distance=0,
           above left=of lecturenumber, anchor=south west, yshift=2mm]
          {\Large CS 111: Operating System Principles};
    \node (title) [inner sep=0, anchor=west, align=right, text=white]
          at (\paperwidth / 6, \paperheight / 2)
          {{\bfseries \Huge \inserttitle{}}};
    \node [inner sep=0, align=right, text=white, node distance=0,
           below right=of title, anchor=north east, yshift=-1mm]
          {{\footnotesize \ttfamily \insertversion}};
    \node [inner sep=0, text=white, align=left, anchor=west]
          (author) at (\paperwidth / 6, \paperheight / 4)
          {\insertauthor};
    \node [text=white, inner sep=0, align=left, node distance=0,
           below left=of author, anchor=north west, yshift=-2mm]
          {\insertdate};
    \node [align=right, anchor=south east, inner sep=2mm, text=white]
          (license) at (\paperwidth, 0)
          {\footnotesize This  work is licensed under a
           \href{http://creativecommons.org/licenses/by-sa/4.0/}
                {\color{uclagold} Creative Commons Attribution-ShareAlike 4.0
                 International License}};
    \node [text=white, inner sep=0, align=right, node distance=0,
           above right=of license, anchor=south east, xshift=-2mm]
          {\Large \ccbysa};
  \end{tikzpicture}
}

\tikzset{
  >=Straight Barb[],
  shorten >=1pt,
  initial text=,
}

\lstset{
  basicstyle=\footnotesize\ttfamily,
}


\lecturenumber{10}
\title{Threads}
\version{2.0.0}
\author{Jon Eyolfson}
\date{July 27, 2021}

\begin{document}
  \begin{frame}[plain, noframenumbering]
    \titlepage
  \end{frame}

  \begin{frame}
    \frametitle{Threads are Like Processes with Shared Memory}

    The same principle as a process, except by default they share memory

    \hspace{2em} They have their own registers, program counter, and stack

    \vspace{2em}

    They have the same address space, so changes appear in each thread

    \vspace{2em}

    You need to explicitly state if any memory is specific to a thread (TLS)
  \end{frame}

  \begin{frame}
    \frametitle{One Process Can have Multiple Threads}

    By default a process just executes code in its own address space

    \vspace{2em}

    Threads allow multiple executions in the same address space

    \vspace{2em}

    They're lighter weight and less expensive to create than processes

    \hspace{2em} They share code, data, file descriptors, etc.
  \end{frame}

  \begin{frame}[fragile]
    \frametitle{Assuming One CPU, Threads Can Express Concurrency}

    A process can appear like it's executing in multiple locations at once

    \hspace{2em} However, the OS is just context switching within a process

    \vspace{2em}

    It may be easier to program concurrently

    \hspace{2em} e.g., handle a web request in a new thread

    \vspace{2em}

    \begin{lstlisting}
while (true) {
  struct request *req = get_request();
  create_thread(process_request, req);
}
    \end{lstlisting}
  \end{frame}

  \begin{frame}
    \frametitle{Threads are Lighter Weight than Processes}

    \begin{tabular}{ll}
      \textbf{Process} & \textbf{Thread} \\
      \\
      Independent code / data / heap & Shared code / data / heap \\
      \\
      Independent execution & Must live within an executing process \\
      \\
      Has its own stack and registers & Has its own stack and registers \\
      \\
      Expensive creation and context switching & Cheap creation and context switching \\
      \\
      Completely removed from OS on exit & Stack removed from process on exit \\
    \end{tabular}

    \vspace{2em}

    When a process dies, all threads within it die as well!
  \end{frame}

  \begin{frame}[fragile]
    \frametitle{We'll be Using POSIX Threads}

    For Windows, there's a Win32 thread, but we're going to use *UNIX threads

    \vspace{2em}

    \lstinline|#include <pthread.h>| --- in your source file

    \vspace{2em}

    \lstinline|-pthread| --- compile and link the pthread library

    \vspace{2em}

    All the pthread functions have documentation in the \lstinline|man| pages
  \end{frame}

  \begin{frame}[fragile]
    \frametitle{You Create Threads with \texttt{pthread\_create}}

    \begin{lstlisting}
int pthread_create(pthread_t* thread, 
                   const pthread_attr_t* attr,
                   void* (*start_routine)(void*),
                   void* arg);
    \end{lstlisting}

    \vspace{1em}

    \begin{tabular}{rl}
      thread & creates a handle to a thread at pointer location \\

      attr & thread attributes (NULL for defaults, more details later) \\

      start\_routine & function to start execution \\

      arg & value to pass to start\_routine \\
    \end{tabular}

    \vspace{2em}
    
    returns 0 on success, error number otherwise (contents of *thread are
    undefined)
  \end{frame}

  \begin{frame}[fragile]
    \frametitle{Creating Threads is a Bit Different than Processes}

    \begin{lstlisting}
#include <pthread.h>
#include <stdio.h>

void* run(void*) {
  printf("In run\n");
  return NULL;
}

int main() {
  pthread_t thread;
  pthread_create(&thread, NULL, &run, NULL);
  printf("In main\n");
}
    \end{lstlisting}

    \vspace{2em}

    What are some differences? Are we missing anything?
  \end{frame}

  \begin{frame}[fragile]
    \frametitle{The \texttt{wait} Equivalent for Threads --- Join}

    \begin{lstlisting}
int pthread_join(pthread_t thread,
                 void** retval)
    \end{lstlisting}

    \vspace{1em}

    \begin{tabular}{rp{10cm}}
  thread & wait for this thread to terminate (thread must be joinable) \\

  retval & stores exit status of thread (set by {\tt pthread\_exit}) to
                 the location pointed by *retval. If cancelled returns
                 {\tt PTHREAD\_CANCELED}. {\tt NULL} is ignored. \\
    \end{tabular}

  \vspace{2em}

  returns 0 on success, error number otherwise 
  
  \vspace{2em}

  {\bf Only call this one time per thread!}

  \hspace{2em} Multiple calls on the same thread leads to undefined behavior
  \end{frame}

  \begin{frame}[fragile]
    \frametitle{Previous Example that Waits Properly}

    \begin{lstlisting}
#include <pthread.h>
#include <stdio.h>

void* run(void*) {
  printf("In run\n");
  return NULL;
}

int main() {
  pthread_t thread;
  pthread_create(&thread, NULL, &run, NULL);
  printf("In main\n");
  pthread_join(thread, NULL);
}
    \end{lstlisting}

    \vspace{2em}

    Now we joined, the thread's resources are cleaned up
  \end{frame}


  \begin{frame}[fragile]
    \frametitle{Ending a Thread Early (Think of \texttt{exit})}

    \begin{lstlisting}
void pthread_exit(void *retval);
    \end{lstlisting}

    \vspace{1em}

    \begin{tabular}{rl}
      retval & return value passed to function that calls {\tt pthread\_join} \\
    \end{tabular}

    \vspace{2em}
    
    Note: start\_routine returning is equivalent of calling {\tt pthread\_exit}

    \hspace{2em} Think of the difference between returning from main and
    \texttt{exit}

    \vspace{2em}
    {\tt pthread\_exit} is called implicitly when the {\tt start\_routine} of a
    thread returns
  \end{frame}

  \begin{frame}[fragile]
    \frametitle{Detached Threads}

    {\it Joinable} threads (the default) wait for someone to call
    {\tt pthread\_join}
  
    \hspace{2em} then they release their resources

    \vspace{2em}

    {\it Detached} threads release their resources when they terminate

    \vspace{2em}

    \begin{lstlisting}
int pthread_detach(pthread_t thread);
    \end{lstlisting}

    \vspace{1em}

    \begin{tabular}{rl}
      thread & marks the thread as detached \\
    \end{tabular}

    \vspace{2em}

    returns 0 on success, error number otherwise

    \vspace{2em}

    Calling {\tt pthread\_detach} on an already detached is undefined
    behavior
  \end{frame}

  \begin{frame}[fragile]
    \frametitle{Detached Threads Aren't Joined}

    \begin{lstlisting}
#include <pthread.h>
#include <stdio.h>

void* run(void*) {
  printf("In run\n");
  return NULL;
}

int main() {
  pthread_t thread;
  pthread_create(&thread, NULL, &run, NULL);
  pthread_detach(thread);
  printf("In main\n");
}
    \end{lstlisting}

    This code just prints ``In main'', why?
  \end{frame}

  \begin{frame}[fragile]
    \frametitle{\texttt{pthread\_exit} in main Waits for All Detached Threads to Finish}

    \begin{lstlisting}
#include <pthread.h>
#include <stdio.h>

void* run(void*) {
  printf("In run\n");
  return NULL;
}

int main() {
  pthread_t thread;
  pthread_create(&thread, NULL, &run, NULL);
  pthread_detach(thread);
  printf("In main\n");
  pthread_exit(NULL);
}
    \end{lstlisting}

    This code now works as expected
  \end{frame}

  \begin{frame}[fragile]
    \frametitle{You Can Use Attributes To Get/Set Thread Variables}

    \begin{lstlisting}
size_t stacksize;
pthread_attr_t attributes;
pthread_attr_init(&attributes);
pthread_attr_getstacksize(&attributes, &stacksize);
printf("Stack size = %i\n", stacksize);
pthread_attr_destroy(&attributes);
    \end{lstlisting}

    \vspace{2em}

    Running this should show a stack size of 8 MiB (on most Linux systems)

    \vspace{2em}

    You can also set a thread state to joinable

    \begin{lstlisting}
pthread_attr_setdetachstate(&attributes,
                            PTHREAD_CREATE_JOINABLE);
    \end{lstlisting}

  \end{frame}

  \begin{frame}
    \frametitle{Multithreading Models}

    Where do we implement threads?

    \hspace{2em} We can either do user or kernel threads

    \vspace{2em}

    User threads are completely in user-space

    \hspace{2em} Kernel doesn't treat your threaded process any differently

    \vspace{2em}

    Kernel threads are implemented in kernel-space

    \hspace{2em} Kernel manages everything for you, and can treat threads specially
  \end{frame}

  \begin{frame}
    \frametitle{Thread Support Requires a Thread Table}

    Similar to the process table we saw previously

    \hspace{2em} It could be in user-space or kernel-space depending

    \vspace{2em}

    For user threads, there also needs to be a run-time system to determine
    scheduling

    \vspace{2em}

    In both models each process can contain multiple threads
  \end{frame}

  \begin{frame}
    \frametitle{We Could Avoid System Calls, or Let a Thread Block Everything}

    For pure user-level threads (again, no kernel support):

    \begin{itemize}
      \item Very fast to create and destroy, no system call, no context switches
      \item One thread the blocks blocks the entire process (kernel can't distinguish)
    \end{itemize}

    \vspace{2em}

    For kernel-level threads:

    \begin{itemize}
      \item Slower, creation involves system calls
      \item If one thread blocks, the kernel can schedule another one
    \end{itemize}
  \end{frame}

  \begin{frame}
    \frametitle{All Threading Libraries You Use Run in User-mode}

    The thread library maps user threads to kernel threads

    \vspace{2em}

    Many-to-one: threads completely implemented in user-space

    \hspace{2em} the kernel only sees one process

    \vspace{2em}

    One-to-one: one user thread maps directly to one kernel thread

    \hspace{2em} the kernel handles everything

    \vspace{2em}

    Many-to-many: many user-level threads map to many kernel level threads
  \end{frame}

  \begin{frame}
    \frametitle{Many-to-one is Pure User-space Implementation}

    It's fast (as outlined before) and portable

    \hspace{2em} It doesn't depend on the system, it's just a library

    \vspace{2em}

    Drawbacks are that one thread blocking causes all threads to block

    \hspace{2em} Also we cannot execute threads in parallel

    \hspace{4em} The kernel will only schedule a process to run
  \end{frame}

  \begin{frame}
    \frametitle{One-to-one Just Uses the Kernel Thread Implementation}

    There's just a thin wrapper around the system calls to make it easier to use

    \vspace{2em}

    Exploits the full parallelism of your machine

    \hspace{2em} The kernel can schedule multiple threads simultaneously

    \vspace{2em}

    We do however need to use a slower system call interface, and we lose some
    control

    \vspace{4em}

    Typically this is the actual implementation used, we'll assume this for
    Linux
  \end{frame}

  \begin{frame}
    \frametitle{Many-to-many is a Hybrid Approach}

    The idea is that there are more user-level threads than kernel-level threads

    \hspace{2em} Cap the number of kernel-level threads to the number we could
    run in parallel

    \vspace{2em}

    We can get the most out of multiple CPUs while reducing the number of system
    calls

    \vspace{2em}

    However, this leads to a complicated thread library

    \hspace{2em} Depending on your mapping luck, you may block other threads
  \end{frame}

  \begin{frame}
    \frametitle{Threads Complicate the Kernel}

    How should \texttt{fork} work with a process with multiple threads?

    \hspace{2em} Copy all threads to the new process, in whatever state they're
    in?

    \hspace{4em} How would this get out of hand?

    \vspace{2em}

    Linux only copies the thread that called \texttt{fork} into a new process

    \hspace{2em} If it hits \texttt{pthread\_exit} it'll always exit with status
    0

    \hspace{4em} (at least as far as I can tell)

    \vspace{2em}

    There's \texttt{pthread\_atfork} (not covered in this course) to control
    what happens
  \end{frame}

  \begin{frame}
    \frametitle{Signals are Sent to a Process}

    Which thread should receive a signal? all of them?

    \vspace{2em}

    Linux will just pick one random thread to handle the signal

    \hspace{2em} Makes concurrency hard, any thread could be interrupted
  \end{frame}

  \begin{frame}
    \frametitle{Instead of Many-to-many, You Can Use a Thread Pool}

    The goal of many-to-many thread mapping is to avoid creation costs

    \vspace{2em}

    A thread pool creates a certain number of threads and a queue of tasks

    \hspace{2em} (maybe as many threads as CPUs in the system)

    \vspace{2em}

    As requests come in, wake them up and give them work to do

    \vspace{2em}

    Reuse them, when there's no work, put them to sleep
  \end{frame}

  \begin{frame}
    \frametitle{Our Next Complication}

    Let's create a program that spawns 8 threads

    \hspace{2em} Each thread increments the same variable 10,000 times

    \vspace{2em}

    What should the final value of the variable be?

    \hspace{2em} The initial value of the variable is 0

    \vspace{2em}

    Run \texttt{examples/lecture-10/pthread-datarace}

    \hspace{2em} Can you fix it?
  \end{frame}

  \begin{frame}
    \frametitle{Both Processes and (Kernel) Threads Enable Parallelization}

    We explored threads, and related them to something we already know (processes)
    \begin{itemize}
      \item Threads are lighter weight, and share memory by default
      \item Each process can have multiple (kernel) threads
      \item Most implementations use one-to-one user-to-kernel thread mapping
      \item The operating system has to manage what happens during a fork, or signals
      \item We now have synchronization issues
    \end{itemize}
  \end{frame}
\end{document}
