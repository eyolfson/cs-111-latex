\documentclass{article}

\usepackage{enumitem}
\usepackage{fontspec}
\usepackage[letterpaper,margin=72pt]{geometry}
\usepackage{hyperref}
\usepackage{import}
\usepackage{listings}
\usepackage{xcolor}

\subimport{../}{colors.tex}

\setsansfont{Overpass}[Scale=MatchLowercase]
\setmonofont{Overpass Mono}[Scale=MatchLowercase]

\renewcommand{\familydefault}{\sfdefault}

\hypersetup{
  colorlinks=true,
  urlcolor=uclablue,
}

\setlist{nosep}

\makeatletter
\newcommand\version[1]{\renewcommand\@version{#1}}
\newcommand\@version{}

\newcommand\labnumber[1]{\renewcommand\@labnumber{#1}}
\newcommand\@labnumber{}

\newcommand\duedate[1]{\renewcommand\@duedate{#1}}
\newcommand\@duedate{}

\renewcommand\maketitle{%
  \noindent
  {\Large \color{uclablue} CS 111: Operating System Principles}

  \noindent
  {\Large \color{uclablue} Lab \@labnumber}\\[-0.75em]

  \noindent
  {\Huge \bfseries \color{uclablue} \@title}
  {\ttfamily \footnotesize \color{uclablue} \@version}\\[-0.75em]

  \noindent
  {\@author}

  \noindent
  {\@date}

  \noindent
  {Due: \@duedate}\\[1em]
}
\makeatother

\lstset{
  basicstyle=\ttfamily,
}


\lecturenumber{16}
\title{Distributed Systems}
\version{2.0.2}
\author{Jon Eyolfson}
\date{August 17, 2021}

\begin{document}
  \begin{frame}[plain, noframenumbering]
    \titlepage
  \end{frame}

  \begin{frame}{Sockets are Another Form of IPC}

    We've seen pipes, shared memory, and signals

    \vspace{2em}

    These forms of IPC assume that the processes are on the same physical
    machine

    \vspace{2em}

    Sockets enable IPC between physical machines, typically over the network
  \end{frame}

  \begin{frame}{Servers Follow 4 Steps to Use Sockets}

    These are all system calls, and have the usual C wrappers:

    \vspace{2em}

    \begin{enumerate}
      \item \texttt{socket}

        \hspace{2em} Create the socket
      \item \texttt{bind}

        \hspace{2em} Attach the socket to some location (a file, IP:port, etc.)
      \item \texttt{listen}

        \hspace{2em} Indicate you're accepting connections, and set the queue limit
      \item \texttt{accept}

        \hspace{2em} Return the next incoming connection for you to handle
    \end{enumerate}
  \end{frame}

  \begin{frame}{Clients Follow 2 Steps to Use Sockets}

    Clients have a much easier time, they use one socket per connection

    \vspace{2em}

    \begin{enumerate}
      \item \texttt{socket}

        \hspace{2em} Create the socket
      \item \texttt{connect}

        \hspace{2em} Connect to some location, the socket can now send/receive
                     data
    \end{enumerate}
  \end{frame}

  \begin{frame}{The \texttt{socket} System Call Sets the Protocol and Type of Socket}

    \lstinline|int socket(int domain, int type, int protocol);|

    \vspace{2em}

    \texttt{domain} is the general protocol, further specified with \texttt{protocol} (mostly unused)

    \hspace{2em} \texttt{AF\_UNIX} is for local communication (on the same physical machine)

    \hspace{2em} \texttt{AF\_INET} is for IPv4 protocol using your network interface

    \hspace{2em} \texttt{AF\_INET6} is for IPv6 protocol using your network interface

    \vspace{2em}

    \texttt{type} is (usually) one of two options: stream or datagram sockets
  \end{frame}

  \begin{frame}{Stream Sockets Use TCP}
    All data sent by a client appears in the same order on the server

    \vspace{2em}

    Forms a persistent connection between client and server

    \vspace{2em}

    Reliable, but may be slow
  \end{frame}

  \begin{frame}{Datagram Sockets Use UDP}
    Sends messages between the client and server

    \vspace{2em}

    No persistent connection between client and server

    \vspace{2em}

    Fast but messages may be reordered, or dropped
  \end{frame}

  \begin{frame}{The \texttt{bind} System Call Sets a Socket to an Address}
    \lstinline|int bind(int socket, const struct sockaddr *address,|

    \hspace{4.4em} \lstinline|socklen_t address_len);|

    \vspace{2em}

    \texttt{socket} is the file descriptor returned from the \texttt{socket}
    system call

    \vspace{2em}

    There's different \lstinline|sockaddr| structures for different protocols

    \hspace{2em} \lstinline|struct sockaddr_un| for local communcation (just a path)

    \hspace{2em} \lstinline|struct sockaddr_in| for IPv4, a IPv4 address
                 (e.g. 8.8.8.8)

    \hspace{2em} \lstinline|struct sockaddr_in6| for IPv6, a IPv6 address
                 (e.g. \lstinline|2001:4860:4860::8888|)
  \end{frame}

  \begin{frame}{The \texttt{listen} System Call Sets Queue Limits for Incoming
                Connections}
    \lstinline|int listen(int socket, int backlog);|

    \vspace{2em}

    \texttt{socket} is still the file descriptor returned from the
    \texttt{socket} system call
    
    \vspace{2em}

    \texttt{backlog} is the limit of the outstanding (not accepted) connections

    \hspace{2em} The kernel manages this queue, and if full will not allow new connections

    \vspace{2em}

    We'll set this to \lstinline|0| to use the default kernel queue size
  \end{frame}

  \begin{frame}{The \texttt{accept} System Call Blocks Until There's a Connection}
    \lstinline|int accept(int socket, struct sockaddr *restrict address,|

    \hspace{5.4em} \lstinline|socklen_t *restrict address_len);|

    \vspace{2em}

    \texttt{socket} is \textit{still} the file descriptor returned from the
    \texttt{socket} system call

    \vspace{2em}

    \texttt{address} and \texttt{address\_len} are locations to write the
    connecting address

    \hspace{2em} Acts as an optional return value, set both to \texttt{NULL} to
                 ignore

    \vspace{2em}

    This returns a new file descriptor, we can \texttt{read} or \texttt{write}
    to as usual
  \end{frame}

  \begin{frame}{The \texttt{connect} System Call Allows a Client to Connect to
                an Address}

    \lstinline|int connect(int sockfd, const struct sockaddr *addr,|
                   
    \hspace{5.9em} \lstinline|socklen_t addrlen);|

    \vspace{2em}

    \texttt{sockfd} is the file descriptor returned by the \texttt{socket}
    system call

    \hspace{2em} The client would need to be using the same protocol and type
    as the server

    \vspace{2em}

    \texttt{addr} and \texttt{addrlen} is the address to connect to, exactly
    like \texttt{bind}

    \vspace{2em}

    If this call succeeds then \texttt{sockfd} is may be used as a normal
    file descriptor
  \end{frame}

  \begin{frame}{Our Example Server Sends ``Hello there!'' to Every Client and
                Disconnects}

    Please see \texttt{examples/lecture-16} in your \texttt{cs111} repository

    \hspace{2em} Relevant source files: \texttt{client.c} and \texttt{server.c}

    \vspace{2em}

    We use a local socket just for demonstration, but you could use IPv4 or IPv6

    \hspace{2em} We use \texttt{example.sock} in the current directory as our
                 socket address

    \vspace{2em}

    Our server uses signals to clean up and terminate from our infinite
    \texttt{accept} loop
  \end{frame}

  \begin{frame}{Instead of \texttt{read}/\texttt{write} There's Also
                \texttt{send}/\texttt{recv} System Calls}

    These system calls are basically the same thing, except they have
    \texttt{flags}

    \vspace{2em}

    Some examples are:

    \hspace{2em} \texttt{MSG\_OOB} --- Send/receive out-of-band data

    \hspace{2em} \texttt{MSG\_PEEK} --- Look at data without reading

    \hspace{2em} \texttt{MSG\_DONTROUTE} --- Send data without routing packets

    \vspace{2em}

    Except for maybe \texttt{MSG\_PEEK}, you do not need to know these

    \vspace{2em}

    \texttt{sento}/\texttt{recvfrom} take an additional address

    \hspace{2em} The kernel ignores the address for stream sockets (there's a
                 connection)
  \end{frame}

  \begin{frame}{Sockets Form a Basis For Distributed Systems}

    You can use a remote procedure call (RPC) to run a function on another
    machine

    \hspace{2em} We send a request to a server, it executes and sends a reply
                 back

    \vspace{2em}

    RPC can be done asynchronously, your process sends a request and doesn't
    block

    \hspace{2em} You can continue working in other threads to keep the process
                 running

    \vspace{2em}

    Easy to program, looks like a normal function running on a server

    \hspace{2em} Details are hidden, converting data to bytes to send (wire
                 format)
  \end{frame}

  \begin{frame}{A RPC Can Fail in More Ways than a Regular Call}

    Maybe a lost packet, interrupted connection, a slow or unresponsive server

    \vspace{2em}

    We don't know what happened on the server:

    \begin{itemize}
      \item It never saw the request
      \item It crashed while executing the request (or just after)
      \item It executed the request, but there was a network error during the
            reply
    \end{itemize}
  \end{frame}

  \begin{frame}{``Best Effort'' RPC Scheme Retries Multiple Times}

    It sends a request, and if there's no response within a timeout, resend 

    \hspace{2em} Try resending a set number of times before giving up and
                 returning an error

    \vspace{2em}

    What if the function increments a variable on the server?

    \hspace{2em} The server could receive the packet after the timeout period

    \hspace{4em} The client resends it and a single call increments twice (or more)!
  \end{frame}

  \begin{frame}{An ``At Most Once'' Scheme is Better}

    The client includes a unique ID with each request

    \hspace{2em} If there's duplicate requests, send the previous reply

    \vspace{2em}

    This is particular useful if the function is not idempotent
    
    \hspace{2em} Idempotent means one or more calls have the same effect

    \vspace{2em}

    You may want something like ``exactly once'', and that's complex (another
    course)
  \end{frame}

  \begin{frame}{File Systems Can Be Networked as Well}

    For example: NFS is a protocol designed to appear as a file system

    \hspace{2em} However, the data is on a server accessed over the network

    \vspace{2em}

    A server accessed using protocols like NFS is called a NAS

    \hspace{2em} NAS stands for Network-attached storage

    \vspace{2em}

    A NAS is typically a single server with a large RAID array
  \end{frame}

  \begin{frame}{You Could Distribute the File System Across Multiple Servers}

    Google File System (GFS) is an example of a distributed file systems

    \vspace{2em}

    It breaks files up into fixed-sized 64 MB chunks and replicates it
    (e.g. 3 times)

    \hspace{2em} Each chunk gets an immutable and unique identifier

    \vspace{2em}

    A single server manages the file to chunk mappings

    \hspace{2em} Clients then access the chunk servers directly
  \end{frame}

  \begin{frame}{GFS Architecture}
    \begin{center}
      \includesvg[height=7cm]{gfs.svg}
    \end{center}
  \end{frame}

  \begin{frame}{GFS Uses Weak Consistency to Scale}

    What about consistency? What if two clients modify a file using
    different servers?

    \hspace{2em} Ideally we'd like the same result as if it was a single server

    \vspace{2em}

    The single server ensures file namespace mutations are atomic

    \hspace{2em} All the metadata is atomic

    \vspace{2em}

    For ``record appends'' (data) is appended ``at least once''

    \hspace{2em} It's up to clients to detect duplicates

    \vspace{2em}

    This system is tailored for large Google scale files
    
    \hspace{2em} We won't go into all the details (there's a SOSP paper if you're
    interested)
  \end{frame}

  \begin{frame}{You Can Distribute Other Tasks Too}

    One machine can only handle so many connections

    \hspace{2em} One machine only has so much bandwidth

    \vspace{2em}

    You can have multiple web servers all serving the same content

    \hspace{2em} You can use a load balancer as your websites entry point

    \hspace{4em} A loader balancer selects the server with the lowest load
  \end{frame}

  \begin{frame}{There Are Limits, Sometimes People Maliciously Overload Systems}

    A denial-of-service attack occurs when you flood a target with bogus
    requests

    \hspace{2em} Sometimes the requests look legitimate, and are hard to detect

    \vspace{2em}

    A distributed denial-of-service (DDoS) occurs from multiple sources

    \hspace{2em} It's even harder to block, often backed by a botnet
  \end{frame}

  \begin{frame}{Distributed Systems Start with Sockets}

    There's networking and distributed systems courses!

    \vspace{2em}

    However, today we learned the basics:

    \begin{itemize}
      \item Sockets are IPC across physical machines
        \begin{itemize}
          \item Sockets require an address (e.g. local and IPv4/IPv6)
          \item There are two types of sockets: stream and datagram
          \item Servers need to bind to an address, listen, and accept connections
          \item Clients need to connect to an address
        \end{itemize}
      \item Networked file systems (NFS)
      \item Distributed file systems (GFS)
      \item Denial-of-service attacks are a unique concern
    \end{itemize}
  \end{frame}
\end{document}
