\documentclass[aspectratio=169]{beamer}

\usepackage{ccicons}
\usepackage{fontspec}
\usepackage{import}
\usepackage{listings}
\usepackage{tikz}

\subimport{../}{colors.tex}

\usetikzlibrary{
  arrows,
  arrows.meta,
  automata,
  backgrounds,
  calc,
  chains,
  decorations.pathreplacing,
  fit,
  matrix,
  overlay-beamer-styles,
  positioning,
  shapes,
  tikzmark,
}
\usetikzmarklibrary{listings}

\hypersetup{
  colorlinks=true,
  urlcolor=uclablue,
}

\setbeamercolor{frametitle}{fg=primarycolor}
\setbeamercolor{structure}{fg=primarycolor}
\setbeamercolor{enumerate item}{fg=black}
\setbeamercolor{itemize item}{fg=black}
\setbeamercolor{itemize subitem}{fg=black}

\setbeamersize{text margin left=26.6mm}
\addtolength{\headsep}{2mm}

\setbeamertemplate{navigation symbols}{}
\setbeamertemplate{headline}{}
\setbeamertemplate{footline}{}
\setbeamertemplate{itemize item}{\color{black}}
\setbeamertemplate{itemize items}[circle]

\setbeamertemplate{footline}{
  \begin{tikzpicture}[remember picture,
                      overlay,
                      shift={(current page.south west)}]
    \node [black!50, inner sep=2mm, anchor=south east]
          at (current page.south east) {\footnotesize \insertframenumber};
  \end{tikzpicture}
}

\setsansfont{Overpass}[Scale=MatchLowercase]
\setmonofont{Overpass Mono}[Scale=MatchLowercase]

\makeatletter
\newcommand\version[1]{\renewcommand\@version{#1}}
\newcommand\@version{}
\def\insertversion{\@version}

\newcommand\lecturenumber[1]{\renewcommand\@lecturenumber{#1}}
\newcommand\@lecturenumber{}
\def\insertlecturenumber{\@lecturenumber}
\makeatother

\setbeamertemplate{title page}
{
  \begin{tikzpicture}[remember picture,
                      overlay,
                      shift={(current page.south west)},
                      background rectangle/.style={fill=uclablue},
                      show background rectangle]
    \node [anchor=west, align=left, inner sep=0, text=white]
          (lecturenumber) at (\paperwidth / 6, \paperheight * 3 / 4)
          {\Large Lecture \insertlecturenumber};
    \node [inner sep=0, align=left, text=white, node distance=0,
           above left=of lecturenumber, anchor=south west, yshift=2mm]
          {\Large CS 111: Operating System Principles};
    \node (title) [inner sep=0, anchor=west, align=right, text=white]
          at (\paperwidth / 6, \paperheight / 2)
          {{\bfseries \Huge \inserttitle{}}};
    \node [inner sep=0, align=right, text=white, node distance=0,
           below right=of title, anchor=north east, yshift=-1mm]
          {{\footnotesize \ttfamily \insertversion}};
    \node [inner sep=0, text=white, align=left, anchor=west]
          (author) at (\paperwidth / 6, \paperheight / 4)
          {\insertauthor};
    \node [text=white, inner sep=0, align=left, node distance=0,
           below left=of author, anchor=north west, yshift=-2mm]
          {\insertdate};
    \node [align=right, anchor=south east, inner sep=2mm, text=white]
          (license) at (\paperwidth, 0)
          {\footnotesize This  work is licensed under a
           \href{http://creativecommons.org/licenses/by-sa/4.0/}
                {\color{uclagold} Creative Commons Attribution-ShareAlike 4.0
                 International License}};
    \node [text=white, inner sep=0, align=right, node distance=0,
           above right=of license, anchor=south east, xshift=-2mm]
          {\Large \ccbysa};
  \end{tikzpicture}
}

\tikzset{
  >=Straight Barb[],
  shorten >=1pt,
  initial text=,
}

\lstset{
  basicstyle=\footnotesize\ttfamily,
}


\labnumber{0}
\title{A Kernel Seedling}
\version{1.0.0}
\author{Jon Eyolfson}
\date{March 29, 2021}
\duedate{April 9, 2021}

\begin{document}

\maketitle

In this lab, you'll setup a virtual machine and write your (probably) first
kernel module.
We'll use VirtualBox as our hypervisor since it supports many different host
operating systems, and is friendly to learn.
You'll be using Git to submit your work and save your progress.
Finally, you'll write a kernel module that adds a file to \lstinline|/proc/|
to expose internal kernel information.

\paragraph{Virtual machine setup.}

After the setup you'll have a fully functioning Linux virtual machine.
You'll also have a shared folder accessible on both the host and guest operating
system.
You're free to edit your files with whatever you're comfortable with in
whichever operating system you wish.
You should only run your code on the virtual machine though.

\begin{enumerate}
  \item Download and Install VirtualBox:
        \url{https://www.virtualbox.org/wiki/Downloads}
  \item Download our virtual machine:
        \url{https://laforge.cs.ucla.edu/cs111/media/cs111/vm.ova}
  \item Import the virtual machine
    \begin{enumerate}
      \item \textit{File} $\rightarrow$ \textit{Import Appliance}
      \item Choose \lstinline|vm.ova| from your local file system
      \item \textit{Next} $\rightarrow$ \textit{Import}
    \end{enumerate}
  \item Create a folder called \textit{Shared} somewhere on your machine
  \item Add the \textit{Shared} folder to the virtual machine
    \begin{enumerate}
      \item Right click \textit{CS 111} from the left panel and hit settings
      \item Click on \textit{Shared Folders} on the left panel
      \item Right click in the right panel and select \textit{Add Shared Folder}
      \item For \textit{Folder Path} select \textit{Other...} and select your
            \textit{Shared} folder on your machine
      \item Leave the \textit{Folder Name} as \textit{Shared}
      \item Check \textit{Auto-mount}
      \item Press \textit{OK}
      \item Press \textit{OK}
    \end{enumerate}
  \item Select \textit{CS 111} from the left panel and click \textit{Start} at
        the top of the right panel
  \item Use \lstinline|cs111| for both the username and password
  \item (Optional) Go to \textit{View} $\rightarrow$ \textit{Virtual Screen 1}
        and resize to any resolution you'd like
\end{enumerate}

\paragraph{Git setup.}
Run all these commands in your \textit{Shared} folder on your virtual machine.
First, open a terminal by going to \textit{Activities} and selecting the
\textit{Terminal} icon on the left.
Your \textit{Shared} directory is mounted at \lstinline|/media/sf_Shared|.
If you're unfamiliar with Git, please check out the
\href{https://git-scm.com/book/en/v2/}{Pro Git} book.
For any of the commands, run them in the terminal.

\begin{enumerate}
  \item Run: \lstinline|git config --global user.name "Your Full Name"|
  \item Run: \lstinline|git config --global user.email your@email.com|
  \item Run: \lstinline|ssh-keygen -o|
    \begin{enumerate}
      \item Press \textit{Enter} for the default location
      \item Press \textit{Enter} for no passphrase
      \item Press \textit{Enter} again to confirm
    \end{enumerate}
  \item Login to the course website
    \begin{enumerate}
      \item (Optional) Go to \textit{Activities} and click the \textit{Firefox}
            icon on the left
    \end{enumerate}
  \item Click your username in the top right
  \item Click \textit{New SSH Key (Text Input)}
  \item Add your SSH key
    \begin{enumerate}
      \item Run: \lstinline|cat ~/.ssh/id_rsa.pub|
      \item Copy and paste the contents into the text box
      \item (Optional) Give the key a comment (it'll be its name)
      \item Press submit
    \end{enumerate}
  
  \item Run: \lstinline|cd /media/sf_Shared|
  \item Run: \lstinline|git clone git@laforge.cs.ucla.edu:spring21/username/cs111|
  \item Run: \lstinline|cd cs111|
  \item Run: \lstinline|git remote add upstream git@laforge.cs.ucla.edu:spring21/jon/cs111|
\end{enumerate}

\paragraph{Lab Setup.}
Ensure you're in the \lstinline|/media/sf_Shared/cs111| directory. Make sure you
have the latest skeleton code from us by running:
\lstinline|git pull upstream main|.
You can finally run: \lstinline|cd lab-00| to begin the lab.

\paragraph{Your task.}

You're going to create a \lstinline|/proc/count| file that shows the current
number of running processes (or tasks) running.
The process table runs within kernel mode, so to access it you'll need to write
a kernel module that runs in kernel mode.
For your submission you'll modify \lstinline|proc_count.c|, and only this file,
for the coding part.
In the \lstinline|lab-00| directory we should be able to run the following
commands:

\begin{lstlisting}[xleftmargin=2em]
make
sudo insmod proc_count.ko
cat /proc/count 
\end{lstlisting}

\noindent
The final command should report a single integer representing the number of
processes (or tasks) running on the machine.

\paragraph{Tips.}

The kernel code is well commented, you can use \url{https://elixir.bootlin.com/}
for looking up functions and macros (symbols). There's already a skeleton that
uses: \lstinline|MODULE_AUTHOR|, \lstinline|MODULE_DESCRIPTION|,
\lstinline|MODULE_LICENSE|, \lstinline|module_init|, \lstinline|module_exit|,
and \lstinline|pr_info|.
You'll probably want to use the following to complete this lab:

\begin{lstlisting}[xleftmargin=2em]
proc_create_single
IS_ERR
PTR_ERR
proc_remove
for_each_process
seq_printf
\end{lstlisting}

\noindent
You can divide this task into small subtasks:

\begin{enumerate}
  \item Properly create and remove \lstinline|/proc/count| when your module
        loads and unloads, respectively
  \item Make \lstinline|/proc/count| return some string when you
        \lstinline|cat /proc/count|
  \item Make \lstinline|/proc/count| return a integer with the number of running
        processes (or tasks) when you \lstinline|cat /proc/count|
\end{enumerate}

\paragraph{Commands.}

You'll have to use the following commands for this lab:

Build your module with \lstinline|make|

Insert your module into the kerenl with \lstinline|sudo insmod proc_count.ko|

Read any information messages printed in the kernel with
\lstinline|sudo dmesg -l info|

Remove your module from the kernel (so you can insert a new one) with
\lstinline|sudo rmmod proc_count|


\end{document}
