\documentclass{article}

\usepackage{enumitem}
\usepackage{fontspec}
\usepackage[letterpaper,margin=72pt]{geometry}
\usepackage{hyperref}
\usepackage{import}
\usepackage{listings}
\usepackage{xcolor}

\subimport{../}{colors.tex}

\setsansfont{Overpass}[Scale=MatchLowercase]
\setmonofont{Overpass Mono}[Scale=MatchLowercase]

\renewcommand{\familydefault}{\sfdefault}

\hypersetup{
  colorlinks=true,
  urlcolor=uclablue,
}

\setlist{nosep}

\makeatletter
\newcommand\version[1]{\renewcommand\@version{#1}}
\newcommand\@version{}

\newcommand\labnumber[1]{\renewcommand\@labnumber{#1}}
\newcommand\@labnumber{}

\newcommand\duedate[1]{\renewcommand\@duedate{#1}}
\newcommand\@duedate{}

\renewcommand\maketitle{%
  \noindent
  {\Large \color{uclablue} CS 111: Operating System Principles}

  \noindent
  {\Large \color{uclablue} Lab \@labnumber}\\[-0.75em]

  \noindent
  {\Huge \bfseries \color{uclablue} \@title}
  {\ttfamily \footnotesize \color{uclablue} \@version}\\[-0.75em]

  \noindent
  {\@author}

  \noindent
  {\@date}

  \noindent
  {Due: \@duedate}\\[1em]
}
\makeatother

\lstset{
  basicstyle=\ttfamily,
}


\lecturenumber{12}
\title{Locking}
\version{2.0.0}
\author{Jon Eyolfson}
\date{August 3, 2021}

\begin{document}
  \begin{frame}[plain, noframenumbering]
    \titlepage
  \end{frame}

  \begin{frame}
    \frametitle{Locks Ensure Mutual Exclusion}

    Only one thread at a time can be between the \texttt{lock} and
    \texttt{unlock} calls

    \vspace{2em}

    It does not help you ensure ordering between threads

    \vspace{2em}

    Assume you had a circular buffer you want to use in a producer/consumer
    scenario

    \hspace{2em} e.g. \texttt{ls | wc}
  \end{frame}

  \begin{frame}[fragile]
    \frametitle{Semaphores are Used for Signaling}

    Semaphores have a {\tt value} that's shared between threads/processes

    \vspace{2em}

    \begin{lstlisting}
#include <semaphore.h>

int sem_init(sem_t *sem, int pshared, unsigned int value)
    \end{lstlisting}

    \vspace{2em}

    There may up to \texttt{value} number of things with the semaphore
    simultaneously

    \vspace{2em}

    It has two fundamental operations {\tt wait} and {\tt post}

    \hspace{2em} \texttt{wait} decrements the value atomically

    \hspace{2em} \texttt{post} increments the value atomically

    \vspace{2em}

    If \texttt{wait} will not return until the value is greater than 0
  \end{frame}

  \begin{frame}[fragile]
    \frametitle{Semaphore API is Similar to \texttt{pthread} Locks}
  
    \begin{lstlisting}
#include <semaphore.h>

int sem_init(sem_t *sem, int pshared, unsigned int value)
int sem_destroy(sem_t *sem);
int sem_post(sem_t *sem);
int sem_wait(sem_t *sem);
int sem_trywait(sem_t *sem);
    \end{lstlisting}

    \vspace{2em}

    All functions return 0 on success

    \vspace{2em}

    The \texttt{pshared} argument is a boolean, you can set it to \texttt{1} for
    IPC

    \hspace{2em} For IPC the semaphore needs to be in shared memory
  \end{frame}

  \begin{frame}[fragile]
    \frametitle{How Could We Make This Print ``Thread 1'' then ``Thread 2''?}

    \begin{lstlisting}
#include <pthread.h>
#include <stdio.h>
#include <semaphore.h>
#include <stdlib.h>

void* p1 (void* arg) { printf("Thread 1\n"); return NULL; }

void* p2 (void* arg) { printf("Thread 2\n"); return NULL; }

int main(int argc, char *argv[])
{
    pthread_t thread[2];
    pthread_create(&thread[0], NULL, p1, NULL);
    pthread_create(&thread[1], NULL, p2, NULL);
    pthread_join(thread[0], NULL);
    pthread_join(thread[1], NULL);
    return EXIT_SUCCESS;
}
    \end{lstlisting}
  \end{frame}

  \begin{frame}[fragile]
    \frametitle{This Code Prints ``Thread 1'' then ``Thread 2''}

    \begin{lstlisting}[escapechar=!]
!\structure{static sem\_t sem;}!

void* p1 (void* arg) {
  printf("Thread 1\n");
  !\structure{sem\_post(\&sem);}!
}

void* p2 (void* arg) {
  !\structure{sem\_wait(\&sem);}!
  printf("Thread 2\n");
}

int main(int argc, char *argv[])
{
  !\structure{sem\_init(\&sem, 0, 0);}!
  /* rest as before */
}
    \end{lstlisting}
  \end{frame}

  \begin{frame}
    \frametitle{No Matter Which Thread Executes First, We Get the Same Order}

    The \texttt{value} is initially 0

    \vspace{2em}

    Assume ``Thread 2'' executes first

    \hspace{2em} It executes \texttt{sem\_wait}, which is 0, and doesn't
    continue

    \vspace{2em}

    ``Thread 1'' doesn't have to wait, it prints first before it increments the
    \texttt{value}

    \vspace{2em}

    ``Thread 2'' can then execute its print statement

    \vspace{2em}

    What happens if we initialized the \texttt{value} to 1?
  \end{frame}

  \begin{frame}
    \frametitle{We Can Use a Semaphore as a Mutex}

    How?
  \end{frame}

  \begin{frame}[fragile]
    \frametitle{Using a Semaphore as a Mutex, Note the \texttt{value}}

    \begin{lstlisting}[escapechar=!]
...
!\structure{static sem\_t sem;}!
static int counter = 0;

void* run(void* arg) {
    for (int i = 0; i < 100; ++i) {
        !\structure{sem\_wait(\&sem);}!
        ++counter;
        !\structure{sem\_post(\&sem);}!
    }
}

int main(int argc, char *argv[])
{
  !\structure{sem\_init(\&sem, 0, 1);}!
  // Create 8 threads
  // Join 8 threads
  printf("counter = %i\n", counter);
}
    \end{lstlisting}
  \end{frame}

  \begin{frame}
    \frametitle{Can We Come Up with a Solution for a Producer/Consumer Problem?}

    Assume you have a circular buffer:

    \vspace{2em}

    \begin{tikzpicture}[>=latex,font=\ttfamily,
      every node/.style={
        minimum width=1.5cm,
        minimum height=1.5em,
        outer sep=0pt,
        draw=black,
        semithick
      }
    ]
      \node at (0,0) (A) {0};
      \node [anchor=west] at (A.east) (B) {1};
      \node [anchor=west] at (B.east) (C) {};
      \node [anchor=west] at (C.east) (D) {};
      \node [anchor=west] at (D.east) (E) {};
      \node [anchor=west] at (E.east) (F) {};
      \node [anchor=west] at (F.east) (G) {n - 1};
      \draw [->,shorten >=2pt,shorten <=2pt,semithick] (G.north) -- +(0,1em) -| (A);

      \node [below=of C, draw=none] (producer) {\sffamily Producer};
      \draw [->, semithick] (producer) -- (C);

      \node [below=of E, draw=none] (consumer) {\sffamily Consumer};
      \draw [->, semithick] (consumer) -- (E);
    \end{tikzpicture}

    \vspace{2em}

    The producer should write to the buffer

    \hspace{2em} As long as the buffer is not full

    \vspace{2em}

    The consumer should read to the buffer

    \hspace{2em} As long as the buffer is not empty
  \end{frame}

  \begin{frame}[fragile]
    \frametitle{We Would Create Two Semaphores, What \texttt{value}s Should We Use?}

    \begin{lstlisting}
sem_t full;
sem_t empty;

sem_init(&full, 0, /* ??? */);
sem_init(&empty, 0, /* ??? */);

void producer() {
  // produce data
  sem_wait(empty);
  // fill a slot
  sem_post(full);
}

void consumer() {
  sem_wait(full);
  // empty a slot
  sem_post(empty);
  // consume data
}
    \end{lstlisting}
  \end{frame}

  \begin{frame}
    \frametitle{The Previous \texttt{value}s Depend on the Buffer Size}

    \texttt{full} should always be initialized to 0

    \vspace{2em}

    \texttt{empty} should be initialized to the size of the buffer ---
    \texttt{N}

    \vspace{2em}

    Do we need any extra locking?

    \onslide<2->{
      \hspace{2em} No, if there's a single producer and consumer

      \hspace{2em} Yes, otherwise
    }
  \end{frame}

  \begin{frame}
    \frametitle{Monitors Are Built Into Some Languages}

    With object oriented programming, developers wanted something easier to use

    \vspace{2em}

    Could mark a method as monitored, and let the compiler handle locking

    \hspace{2em} An object can only have one thread active in its monitored
    methods

    \vspace{2em}

    It's basically one mutex per object, created for you

    \hspace{2em} The compiler inserts calls to lock and unlock
  \end{frame}

  \begin{frame}[fragile]
    \frametitle{Java's \texttt{synchronized} Keyword is an Example of a Monitor}

    \begin{lstlisting}
public class Account {
  int balance;
  public synchronized void deposit(int amount)  { balance += amount; }
  public synchronized void withdraw(int amount) { balance -= amount; }
}
    \end{lstlisting}

    the compiler transforms to:

    \begin{lstlisting}[escapechar=!]
  public void deposit(int amount) {
    !\structure{lock(this.monitor);}!
    balance += amount;
    !\structure{unlock(this.monitor);}!
  }
  public void withdraw(int amount) {
    !\structure{lock(this.monitor);}!
    balance -= amount;
    !\structure{unlock(this.monitor);}!
  }
    \end{lstlisting}
  \end{frame}

  \begin{frame}[fragile]
    \frametitle{Condition Variables Behave Like Semaphores}

    You can create your own custom queue of threads

    \vspace{2em}

    \begin{lstlisting}
#include <pthread.h>

int pthread_cond_init(pthread_cond_t *cond,
                      const pthread_condattr_t *attr)
int pthread_cond_destroy(pthread_cond_t *cond);
int pthread_cond_signal(pthread_cond_t *cond);
int pthread_cond_broadcast(pthread_cond_t *cond);
int pthread_cond_wait(pthread_cond_t *cond, pthread_mutex_t *mutex);
int pthread_cond_timedwait(pthread_cond_t *cond, pthread_mutex_t *mutex,
                           const struct timespec *abstime);
    \end{lstlisting}

    \vspace{2em}

    The \texttt{wait} functions add this thread to the queue

    \hspace{2em} \texttt{signal} wakes up one thread, \texttt{broadcast} wakes
    up all threads
  \end{frame}

  \begin{frame}
    \frametitle{Condition Variables MUST Be Paired with a Mutex}

    Any calls to \texttt{wait}, \texttt{signal}, and \texttt{broadcast} must
    already hold the mutex

    \vspace{2em}

    Why? \texttt{wait} needs to add itself to the queue safely (without data
    races)

    \hspace{2em} It needs the mutex as an argument to unlock it before going to
    sleep

    \vspace{2em}

    One mutex can also protect multiple condition variables

    \vspace{2em}

    We'll only consider calls to \texttt{wait} and \texttt{signal}
  \end{frame}

  \begin{frame}[fragile]
    \frametitle{We Can Use Condition Variables for Our Producer/Consumer}

    \begin{columns}
      \begin{column}{0.5\textwidth}
        \begin{lstlisting}
pthread_mutex_t mutex;
int nfilled;
pthread_cond_t has_filled;
pthread_cond_t has_empty;

void producer() {
  // produce data
  pthread_mutex_lock(&mutex);
  if (nfilled == N) {
    pthread_cond_wait(&has_empty,
                      &mutex);
  }
  // fill a slot
  ++nfilled;
  pthread_cond_signal(&has_filled);
  pthread_mutex_unlock(&mutex);
}
        \end{lstlisting}
      \end{column}
      \begin{column}{0.5\textwidth}
        \begin{lstlisting}
void consumer() {
  pthread_mutex_lock(&mutex);
  if (nfilled == 0) {
    pthread_cond_wait(&has_filled,
                      &mutex);
  }
  // empty a slot
  --nfilled;
  pthread_cond_signal(&has_empty);
  pthread_mutex_unlock(&mutex);
  // consume data
}
        \end{lstlisting}
      \end{column}
    \end{columns}
  \end{frame}

  \begin{frame}
    \frametitle{Condition Variables Serve a Similar Purpose as Semaphores}

    You can think of semaphores as a special case of condition variables

    \hspace{2em} They'll go to sleep when the value is 0, when it's greater
    than 0 they wake up

    \vspace{2em}

    You can implement one using the other, however it can get messy

    \vspace{2em}

    For complex conditions condition variables offer much better clarity
  \end{frame}

  \begin{frame}
    \frametitle{Locking Granularity is the Extent of Your Locks}

    You need locks to prevent data races

    \vspace{2em}

    Lock large sections of your program, or divide the locks and
    use smaller sections?

    \hspace{2em} Lab 3

    \vspace{2em}
    
    Things to consider about locks:

    \begin{itemize}
     \item Overhead
      \item Contention
      \item Deadlocks
    \end{itemize}
  \end{frame}

  \begin{frame}
    \frametitle{Locking Overheads}

    \begin{itemize}
      \item Memory allocated
      \item Initialization and destruction time
      \item Time to acquire and release locks
    \end{itemize}

    \vspace{2em}

    The more locks you have, the greater each cost is going to be
  \end{frame}

  \begin{frame}
    \frametitle{You Do NOT Want Deadlocks}

    The more locks you have, the more you have to worry about deadlocks

    \vspace{2em}

    Conditions for deadlocking:

    \begin{enumerate}
      \item Mutual Exclusion (of course for simple locks)
      \item Hold and Wait (you have a lock and try to acquire another)
      \item No Preemption (we can't take simple locks away)
      \item Circular Wait (waiting for a lock held by another process)
    \end{enumerate}
  \end{frame}

  \begin{frame}[containsverbatim]
    \frametitle{A Simple Deadlock Example}

    Consider two processors trying to get two {\it locks}:

    \vspace{2em}

    \begin{columns}
      \column{0.4\textwidth}
        {\bf Thread 1}

        \verb+Get Lock 1+

        \verb+Get Lock 2+

        \verb+Release Lock 2+

        \verb+Release Lock 1+
      \column{0.4\textwidth}
        {\bf Thread 2}

        \verb+Get Lock 2+

        \verb+Get Lock 1+

        \verb+Release Lock 1+

        \verb+Release Lock 2+
    \end{columns}

    \vspace{2em}

    Thread 1 gets Lock 1, then Thread 2 gets Lock 2, now
    they both wait for each other

    \hspace{2em} Deadlock
  \end{frame}

  \begin{frame}[fragile]
    \frametitle{You Can Ensure Order to Prevent Deadlocks}

    \begin{lstlisting}
void f1() {
    locktype_lock(&l1);
    locktype_lock(&l2);
    // protected code
    locktype_unlock(&l2);
    locktype_unlock(&ll);    
}
    \end{lstlisting}

    This code will not deadlock, you can only get {\tt l2} if you have
    {\tt l1}
  \end{frame}

  \begin{frame}[fragile]
    \frametitle{You Could Also Prevent A Deadlock by Using {\tt trylock}}

    Remember, for {\tt pthread} there's {\tt trylock} that returns 0 if it gets
    the lock

    \begin{lstlisting}
void f2() {
    locktype_lock(&l1);
    while (locktype_trylock(&l2) != 0) {
        locktype_unlock(&l1);
        // wait
        locktype_lock(&l1);
    }
    // protected code
    locktype_unlock(&l2);
    locktype_unlock(&ll);    
}
    \end{lstlisting}

    This code will not deadlock, it will give up {\tt l1} if it can't get
    {\tt l2}
  \end{frame}

  \begin{frame}
    \frametitle{We Explored More Advanced Locking}

    Before we did mutual exclusion, now we can ensure order

    \begin{itemize}
      \item Semaphores are an atomic value that can be used for signaling
      \item Condition variables are clearer for complex condition signaling
      \item Locking granularity matters, you'll find out in Lab 3
      \item You must prevent deadlocks
    \end{itemize}
  \end{frame}
\end{document}
