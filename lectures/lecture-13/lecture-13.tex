\documentclass{article}

\usepackage{enumitem}
\usepackage{fontspec}
\usepackage[letterpaper,margin=72pt]{geometry}
\usepackage{hyperref}
\usepackage{import}
\usepackage{listings}
\usepackage{xcolor}

\subimport{../}{colors.tex}

\setsansfont{Overpass}[Scale=MatchLowercase]
\setmonofont{Overpass Mono}[Scale=MatchLowercase]

\renewcommand{\familydefault}{\sfdefault}

\hypersetup{
  colorlinks=true,
  urlcolor=uclablue,
}

\setlist{nosep}

\makeatletter
\newcommand\version[1]{\renewcommand\@version{#1}}
\newcommand\@version{}

\newcommand\labnumber[1]{\renewcommand\@labnumber{#1}}
\newcommand\@labnumber{}

\newcommand\duedate[1]{\renewcommand\@duedate{#1}}
\newcommand\@duedate{}

\renewcommand\maketitle{%
  \noindent
  {\Large \color{uclablue} CS 111: Operating System Principles}

  \noindent
  {\Large \color{uclablue} Lab \@labnumber}\\[-0.75em]

  \noindent
  {\Huge \bfseries \color{uclablue} \@title}
  {\ttfamily \footnotesize \color{uclablue} \@version}\\[-0.75em]

  \noindent
  {\@author}

  \noindent
  {\@date}

  \noindent
  {Due: \@duedate}\\[1em]
}
\makeatother

\lstset{
  basicstyle=\ttfamily,
}


\lecturenumber{13}
\title{Memory Allocation}
\version{2.0.0}
\author{Jon Eyolfson}
\date{August 5, 2021}

\begin{document}
  \begin{frame}[plain, noframenumbering]
    \titlepage
  \end{frame}

  \begin{frame}
    \frametitle{Static Allocation is the Simplest Strategy}

    Create a fixed size allocation in your program

    \hspace{2em} e.g. \lstinline|char buffer[4096];|

    \vspace{2em}

    When the program loads, the kernel sets aside that memory
    for you

    \vspace{2em}

    That memory exists as long as your process does, no need to free
  \end{frame}

  \begin{frame}
    \frametitle{Dynamic Allocation is Often Required}

    You may only conditionally require memory

    \hspace{2em} Static allocations are sometimes wasteful

    \vspace{2em}

    You may not know the size of the allocation

    \hspace{2em} Static allocations need to account for the maximum size

    \vspace{2em}

    Where do you allocate memory?

    \hspace{2em} You can either allocate on the stack or on the heap
  \end{frame}

  \begin{frame}
    \frametitle{Stack Allocation is Mostly Done for You in C}

    Think of normal variables

    \hspace{2em} e.g. \lstinline|int x;|

    \vspace{2em}

    The compiler internally inserts \lstinline|alloca| calls

    \hspace{2em} e.g. \lstinline|int *px = (int*) alloca(4);|

    \vspace{2em}

    Whenever the function that called \lstinline|alloca| returns, it
    frees all the memory

    \hspace{2em} This just restores the previous stack pointer

    \vspace{2em}

    This won't work if you try to use the memory after returning
  \end{frame}

  \begin{frame}
    \frametitle{You've Used Dynamic Allocation Before}

    These are the \lstinline|malloc| family of functions

    \vspace{2em}

    The most flexible way to use memory, but is also the most difficult to get
    right

    \vspace{2em}

    You have to properly handle your memory lifetimes, and \lstinline|free|
    exactly once

    \vspace{2em}

    Also, there's a new concern --- fragmentation
  \end{frame}

  \begin{frame}
    \frametitle{Fragmentation is a Unique Issue for Dynamic Allocation}

    You allocate memory in different sized contiguous blocks

    \hspace{2em} Compaction is not possible and every allocation decision is
    permanent

    \vspace{2em}

    A fragment is a small contiguous block of memory that cannot handle an allocation

    \hspace{2em} You can think of it as a ``hole'' in memory, wasting space

    \vspace{2em}

    There are 3 requirements for fragmentation
    \begin{enumerate}
        \item Different allocation lifetimes
        \item Different allocation sizes
        \item Inability to relocate previous allocations  
    \end{enumerate}
  \end{frame}

  \begin{frame}{There's Internal and External Fragmentation}

    External fragmentation occurs when you allocate different sized blocks

    \hspace{2em} There's no room for an allocation between the blocks

    \begin{center}
    \begin{tikzpicture}[node distance=0mm and 0mm]
      \node[draw,rectangle,minimum width=50,minimum height=30,fill=red]
        (a0) {};
      \node[draw,rectangle,minimum width=30,minimum height=30]
        (a1) [right=of a0] {};
      \node[draw,rectangle,minimum width=20,minimum height=30,fill=red]
        (a2) [right=of a1] {};
      \node[draw,rectangle,minimum width=60,minimum height=30]
        (a3) [right=of a2] {};
    \end{tikzpicture}
    \end{center}

    Internal fragmentation occurs when you allocate fixed sized blocks

    \hspace{2em} There's wasted space within a block

    \begin{center}
    \begin{tikzpicture}[node distance=0mm and 0mm]
      \node[draw,rectangle,minimum width=50,minimum height=20,fill=red]
        (a0) {};
      \node[draw,rectangle,minimum width=30,minimum height=20,fill=black]
        (a1) [right=of a0] {};
      \node[draw,rectangle,minimum width=20,minimum height=20,fill=red]
        (a2) [right=of a1] {};
      \node[draw,rectangle,minimum width=60,minimum height=20,fill=black]
        (a3) [right=of a2] {};
      \node[draw,rectangle,minimum width=80,minimum height=30,xshift=15,yshift=25]
        [below=of a0] {};
      \node[draw,rectangle,minimum width=80,minimum height=30,xshift=30,yshift=25]
        [below=of a2] {};
    \end{tikzpicture}
    \end{center}

    \begin{flushright}
      Credit: \href{https://git.scc.kit.edu/uurqi/os-tutorium}{Daniel Ritz}
    \end{flushright}
  \end{frame}

  \begin{frame}
    \frametitle{We Want to Minimize Fragmentation}

    Fragmentation is just wasted space, which we should prevent

    \vspace{2em}

    We want to reduce the number of ``holes'' between blocks of memory

    \hspace{2em} If we have holes, we'd like to keep them as large as possible

    \vspace{2em}

    Our goal is to keep allocating memory without wasting space
  \end{frame}

  \begin{frame}
    \frametitle{Allocator Implementations Usually Use a Free List}

    They keep track of free blocks of memory by chaining them together

    \hspace{2em} Implemented with a linked list

    \vspace{2em}

    We need to be able to handle a request of any size

    \vspace{2em}

    For allocation, we choose a block large enough for the request

    \hspace{2em} Remove it from the free list

    \vspace{2em}

    For deallocation, we add the block back to the free list

    \hspace{2em} If it's adjacent to another free block, we can merge them
  \end{frame}

  \begin{frame}
    \frametitle{There are 3 General Heap Allocation Strategies}

    Best fit: choose the smallest block that can satisfy the request

    \hspace{2em} Needs to search through the whole list (unless there's an
    exact match)

    \vspace{2em}

    Worst fit: choose largest block (most leftover space)

    \hspace{2em} Also has to search through the list

    \vspace{2em}
    
    First fit: choose first block that can satisfy request
  \end{frame}

  \begin{frame}{Allocating Using Best Fit (1)}

    Note that blocks with a blank background and a number are free

    \vspace{2em}

    \begin{tikzpicture}[node distance=0mm and 0mm]
      \node[draw,rectangle,minimum width=40,minimum height=20,fill=solarizedred]
        (a0) {};
      \node[draw,rectangle,minimum width=100,minimum height=20]    (a1)  [right=of a0]      {100};
      \node[draw,rectangle,minimum width=120,minimum height=20,fill=solarizedblue]    (a2) [right=of a1]       {};
      \node[draw,rectangle,minimum width=60,minimum height=20]    (a3)    [right=of a2]    {60};

      \node[draw,rectangle,minimum width=40,minimum height=20,yshift=-20,fill=solarizedgreen]
        [below=of a0] {40};
    \end{tikzpicture}

    Where do we allocate this block?
  \end{frame}

  \begin{frame}{Allocating Using Best Fit (2)}
    \begin{tikzpicture}[node distance=0mm and 0mm]
            
      \node[draw,rectangle,minimum width=40,minimum height=20,fill=solarizedred]    (a0)        {};
      \node[draw,rectangle,minimum width=100,minimum height=20]    (a1)  [right=of a0]      {100};
      \node[draw,rectangle,minimum width=120,minimum height=20,fill=solarizedblue]    (a2) [right=of a1]       {};
      \node[draw,rectangle,minimum width=40,minimum height=20,fill=solarizedgreen]    (a3)    [right=of a2]    {};
      \node[draw,rectangle,minimum width=20,minimum height=20]    (a4)    [right=of a3]    {20};

      \node[draw,rectangle,minimum width=60,minimum height=20,yshift=-20,xshift=10,fill=solarizedpurple] [below=of a0] {60};

    \end{tikzpicture}

    Where do we allocate this block?
  \end{frame}

  \begin{frame}{Allocating Using Best Fit (3)}
    \begin{tikzpicture}[node distance=0mm and 0mm]
      \node[draw,rectangle,minimum width=40,minimum height=20,fill=solarizedred]    (a0)        {};
      \node[draw,rectangle,minimum width=60,minimum height=20,fill=solarizedpurple]    (a1)  [right=of a0]      {};
      \node[draw,rectangle,minimum width=40,minimum height=20]    (a2)  [right=of a1]      {40};
      \node[draw,rectangle,minimum width=120,minimum height=20,fill=solarizedblue]    (a3) [right=of a2]       {};
      \node[draw,rectangle,minimum width=40,minimum height=20,fill=solarizedgreen]    (a4)    [right=of a3]    {};
      \node[draw,rectangle,minimum width=20,minimum height=20]    (a5)    [right=of a4]    {20};

      \node[draw,rectangle,minimum width=60,minimum height=20,yshift=-20,xshift=10,fill=solarizedmagenta] [below=of a0] {60};
    \end{tikzpicture}

    The next block does not fit anywhere
  \end{frame}

  \begin{frame}{Allocating Using Worst Fit (1)}
    \begin{tikzpicture}[node distance=0mm and 0mm]
        \node[draw,rectangle,minimum width=40,minimum height=20,fill=solarizedred]    (a0)        {};
        \node[draw,rectangle,minimum width=100,minimum height=20]    (a1)  [right=of a0]      {100};
        \node[draw,rectangle,minimum width=120,minimum height=20,fill=solarizedblue]    (a2) [right=of a1]       {};
        \node[draw,rectangle,minimum width=60,minimum height=20]    (a3)    [right=of a2]    {60};

        \node[draw,rectangle,minimum width=40,minimum height=20,yshift=-20,fill=solarizedgreen] [below=of a0] {40};
    \end{tikzpicture}

    Where do we allocate this block?
  \end{frame}

  \begin{frame}{Allocating Using Worst Fit (2)}
    \begin{tikzpicture}[node distance=0mm and 0mm]
        \node[draw,rectangle,minimum width=40,minimum height=20,fill=solarizedred]    (a0)        {};
        \node[draw,rectangle,minimum width=40,minimum height=20,fill=solarizedgreen]    (a1)  [right=of a0]      {};
        \node[draw,rectangle,minimum width=60,minimum height=20]    (a2)    [right=of a1]    {60};
        \node[draw,rectangle,minimum width=120,minimum height=20,fill=solarizedblue]    (a3) [right=of a2]       {};
        \node[draw,rectangle,minimum width=60,minimum height=20]    (a4)    [right=of a3]    {60};

        \node[draw,rectangle,minimum width=60,minimum height=20,yshift=-20,xshift=10,fill=solarizedpurple] [below=of a0] {60};
    \end{tikzpicture}

    Where do we allocate this block?
  \end{frame}

  \begin{frame}{Allocating Using Worst Fit (3)}
    \begin{tikzpicture}[node distance=0mm and 0mm]
        \node[draw,rectangle,minimum width=40,minimum height=20,fill=solarizedred]    (a0)        {};
        \node[draw,rectangle,minimum width=40,minimum height=20,fill=solarizedgreen]    (a1)  [right=of a0]      {};
        \node[draw,rectangle,minimum width=60,minimum height=20,fill=solarizedpurple]    (a2)    [right=of a1]    {};
        \node[draw,rectangle,minimum width=120,minimum height=20,fill=solarizedblue]    (a3) [right=of a2]       {};
        \node[draw,rectangle,minimum width=60,minimum height=20]    (a4)    [right=of a3]    {60};

        \node[draw,rectangle,minimum width=60,minimum height=20,yshift=-20,xshift=10,fill=solarizedmagenta] [below=of a0] {60};
    \end{tikzpicture}

    Next block fits exactly in remaining space
  \end{frame}

  \begin{frame}
    \frametitle{Best Fit and Worst Fit are Both Slow}

    Best fit: tends to leave very large holes and very small holes

    \hspace{2em} Small holes may be useless

    \vspace{2em}

    Worst fit: simulation says it's the worst in terms of storage utilization

    \vspace{2em}

    First fit: tends to leave ``average'' size holes
  \end{frame}

  \begin{frame}
    \frametitle{The Buddy Allocator Restricts the Problem}

    Typically allocation requests are of size $\mathsf{2^n}$

    \hspace{2em} e.g. 2, 4, 8, 16, 32, ..., 4096, ...

    \vspace{2em}

    Restrict allocations to be powers of 2 to enable a more efficient
    implementation

    \hspace{2em} Split blocks into 2 until you can handle the request

    \vspace{2em}

    We want to be able to do fast searching and merging
  \end{frame}

  \begin{frame}
    \frametitle{You Can Implement the Buddy Allocator Using Multiple Lists}
    

    We restrict the requests to be $\mathsf{2^k, 0 \leq k \leq N}$ (round up if needed)

    \vspace{2em}

    Our implementation would use $\mathsf{N+1}$ free lists of blocks for each
    size

    \hspace{2em}

    For a request of size $\mathsf{2^k}$, we search the free list until we find
    a big enough block

    \hspace{2em} Search $\mathsf{k, k+1, k+2, ...}$ until we find one

    \hspace{4em} Recursively divide the block if needed until it's the correct size

    \hspace{6em} Insert “buddy” blocks into free lists

    \vspace{2em}

    For deallocations, we coalesce the buddy blocks back together

    \hspace{2em} Recursively coalesce the blocks if needed
  \end{frame}

  \begin{frame}{Using the Buddy Allocator (1)}

    \begin{tikzpicture}[node distance=5mm and 0mm]
            
        \node[draw,rectangle,minimum width=280,minimum height=20,fill=black,text=white]    (a00)        {256};
        \node[draw,rectangle,minimum width=140,minimum height=20,fill=black,text=white,xshift=-70]    (a10) [below=of a00]       {128};
        \node[draw,rectangle,minimum width=140,minimum height=20,fill=black,text=white]    (a11) [right=of a10]       {128};
        \node[draw,rectangle,minimum width=70,minimum height=20,fill=black,text=white,xshift=-35]    (a20) [below=of a10]       {64};
        \node[draw,rectangle,minimum width=70,minimum height=20]    (a21) [right=of a20]       {64};
        \node[draw,rectangle,minimum width=70,minimum height=20,fill=solarizedblue,xshift=-35]    (a22) [below=of a11]       {64};
        \node[draw,rectangle,minimum width=70,minimum height=20]    (a23) [right=of a22]       {64};
        \node[draw,rectangle,minimum width=35,minimum height=20,fill=solarizedred,xshift=-17]    (a30) [below=of a20]       {32};
        \node[draw,rectangle,minimum width=35,minimum height=20]    (a31) [right=of a30]       {32};

        \draw[->] (a00.south) -- (a10.north);
        \draw[->] (a00.south) -- (a11.north);
        \draw[->] (a10.south) -- (a20.north);
        \draw[->] (a10.south) -- (a21.north);
        \draw[->] (a11.south) -- (a22.north);
        \draw[->] (a11.south) -- (a23.north);
        \draw[->] (a20.south) -- (a30.north);
        \draw[->] (a20.south) -- (a31.north);

    \end{tikzpicture}

    \vspace{2em}

    Where do we allocate a request of size 28?
  \end{frame}

  \begin{frame}{Using the Buddy Allocator (2)}

    \begin{tikzpicture}[node distance=5mm and 0mm]
            
        \node[draw,rectangle,minimum width=280,minimum height=20,fill=black,text=white]    (a00)        {256};
        \node[draw,rectangle,minimum width=140,minimum height=20,fill=black,text=white,xshift=-70]    (a10) [below=of a00]       {128};
        \node[draw,rectangle,minimum width=140,minimum height=20,fill=black,text=white]    (a11) [right=of a10]       {128};
        \node[draw,rectangle,minimum width=70,minimum height=20,fill=black,text=white,xshift=-35]    (a20) [below=of a10]       {64};
        \node[draw,rectangle,minimum width=70,minimum height=20]    (a21) [right=of a20]       {64};
        \node[draw,rectangle,minimum width=70,minimum height=20,fill=solarizedblue,xshift=-35]    (a22) [below=of a11]       {64};
        \node[draw,rectangle,minimum width=70,minimum height=20]    (a23) [right=of a22]       {64};
        \node[draw,rectangle,minimum width=35,minimum height=20,fill=solarizedred,xshift=-17]    (a30) [below=of a20]       {32};
        \node[draw,rectangle,minimum width=35,minimum height=20,fill=solarizedgreen]    (a31) [right=of a30]       {32};

        \draw[->] (a00.south) -- (a10.north);
        \draw[->] (a00.south) -- (a11.north);
        \draw[->] (a10.south) -- (a20.north);
        \draw[->] (a10.south) -- (a21.north);
        \draw[->] (a11.south) -- (a22.north);
        \draw[->] (a11.south) -- (a23.north);
        \draw[->] (a20.south) -- (a30.north);
        \draw[->] (a20.south) -- (a31.north);

    \end{tikzpicture}

    \vspace{2em}

    Where do we allocate a request of size 32?
  \end{frame}

  \begin{frame}{Using the Buddy Allocator (3)}

    \begin{tikzpicture}[node distance=5mm and 0mm]
            
        \node[draw,rectangle,minimum width=280,minimum height=20,fill=black,text=white]    (a00)        {256};
        \node[draw,rectangle,minimum width=140,minimum height=20,fill=black,text=white,xshift=-70]    (a10) [below=of a00]       {128};
        \node[draw,rectangle,minimum width=140,minimum height=20,fill=black,text=white]    (a11) [right=of a10]       {128};
        \node[draw,rectangle,minimum width=70,minimum height=20,fill=black,text=white,xshift=-35]    (a20) [below=of a10]       {64};
        \node[draw,rectangle,minimum width=70,minimum height=20,fill=black,text=white]    (a21) [right=of a20]       {64};
        \node[draw,rectangle,minimum width=70,minimum height=20,fill=solarizedblue,xshift=-35]    (a22) [below=of a11]       {64};
        \node[draw,rectangle,minimum width=70,minimum height=20]    (a23) [right=of a22]       {64};
        \node[draw,rectangle,minimum width=35,minimum height=20,fill=solarizedred,xshift=-17]    (a30) [below=of a20]       {32};
        \node[draw,rectangle,minimum width=35,minimum height=20,fill=solarizedgreen]    (a31) [right=of a30]  {32};
        \node[draw,rectangle,minimum width=35,minimum height=20,fill=solarizedpurple,xshift=-17]    (a32) [below=of a21]       {32};
        \node[draw,rectangle,minimum width=35,minimum height=20,]    (a33) [right=of a32]       {32};

        \draw[->] (a00.south) -- (a10.north);
        \draw[->] (a00.south) -- (a11.north);
        \draw[->] (a10.south) -- (a20.north);
        \draw[->] (a10.south) -- (a21.north);
        \draw[->] (a11.south) -- (a22.north);
        \draw[->] (a11.south) -- (a23.north);
        \draw[->] (a20.south) -- (a30.north);
        \draw[->] (a20.south) -- (a31.north);
        \draw[->] (a21.south) -- (a32.north);
        \draw[->] (a21.south) -- (a33.north);

    \end{tikzpicture}

    \vspace{2em}

    What happens when the we free the size 64 block?
\end{frame}


  \begin{frame}{Using the Buddy Allocator (4)}

    \begin{tikzpicture}[node distance=5mm and 0mm]
            
        \node[draw,rectangle,minimum width=280,minimum height=20,fill=black,text=white]    (a00)        {256};
        \node[draw,rectangle,minimum width=140,minimum height=20,fill=black,text=white,xshift=-70]    (a10) [below=of a00]       {128};
        \node[draw,rectangle,minimum width=140,minimum height=20]    (a11) [right=of a10]       {128};
        \node[draw,rectangle,minimum width=70,minimum height=20,fill=black,text=white,xshift=-35]    (a20) [below=of a10]       {64};
        \node[draw,rectangle,minimum width=70,minimum height=20,fill=black,text=white]    (a21) [right=of a20]       {64};
        \node[draw,rectangle,minimum width=35,minimum height=20,fill=solarizedred,xshift=-17]    (a30) [below=of a20]       {32};
        \node[draw,rectangle,minimum width=35,minimum height=20,fill=solarizedgreen]    (a31) [right=of a30]  {32};
        \node[draw,rectangle,minimum width=35,minimum height=20,fill=solarizedpurple,xshift=-17]    (a32) [below=of a21]       {32};
        \node[draw,rectangle,minimum width=35,minimum height=20,]    (a33) [right=of a32]       {32};

        \draw[->] (a00.south) -- (a10.north);
        \draw[->] (a00.south) -- (a11.north);
        \draw[->] (a10.south) -- (a20.north);
        \draw[->] (a10.south) -- (a21.north);
        \draw[->] (a20.south) -- (a30.north);
        \draw[->] (a20.south) -- (a31.north);
        \draw[->] (a21.south) -- (a32.north);
        \draw[->] (a21.south) -- (a33.north);

    \end{tikzpicture}

  \end{frame}

  \begin{frame}
    \frametitle{Buddy Allocators are Used in Linux}


    Advantages

    \hspace{2em} Fast and simple compared to general dynamic memory allocation

    \hspace{2em} Avoids external fragmentation by keeping free physical pages contiguous

    \vspace{2em}

    Disadvantages

    \hspace{2em} There's always internal fragmentation

    \hspace{4em} We always round up the allocation size if it's not a power of 2
  \end{frame}

  \begin{frame}{Slab Allocators Take Advantage of Fixed Size Allocations}

  Allocate objects of same size from a dedicated pool

  \hspace{2em} All structures of the same type are the same size

  \vspace{2em}

  Every object type has it's own pool with blocks of the correct size

    \hspace{2em} This prevents internal fragmentation
  \end{frame}

  \begin{frame}
    \frametitle{Slab is a Cache of ``Slots''}

    Each allocation size has a corresponding slab of slots (one slot holds one
    allocation)

    \vspace{2em}

    Instead of a linked list, we can use a bitmap (there's a mapping between
    bit and slot)

    \hspace{2em} For allocations we set the bit and return the slot

    \hspace{2em} For deallocations we just clear the bit

    \vspace{2em}

    The slab an be implemented on top of the buddy allocator
  \end{frame}

  \begin{frame}{Each Slab Can Be Allocated using the Buddy Allocator}
    
    Consider two object sizes: A and B

    \vspace{2em}
    
    \begin{tikzpicture}[node distance=0mm and 0mm]

        \node[draw,rectangle,minimum width=20,minimum height=20,fill=solarizedred]    (a0)        {};
        \node[draw,rectangle,minimum width=20,minimum height=20]    (a1)     [right=of a0]   {};
        \node[draw,rectangle,minimum width=20,minimum height=20]    (a2)     [right=of a1]   {};
        \node[draw,rectangle,minimum width=20,minimum height=20,fill=solarizedred]    (a3)     [right=of a2]   {};
        \node[draw,rectangle,minimum width=8,minimum height=20,fill=black]    (a4)     [right=of a3]   {};

        \node[draw,rectangle,minimum width=20,minimum height=20]    (a5)     [right=of a4]   {};
        \node[draw,rectangle,minimum width=20,minimum height=20]    (a6)     [right=of a5]   {};
        \node[draw,rectangle,minimum width=20,minimum height=20]    (a7)     [right=of a6]   {};
        \node[draw,rectangle,minimum width=20,minimum height=20]    (a8)     [right=of a7]   {};
        \node[draw,rectangle,minimum width=8,minimum height=20,fill=black]    (a9)     [right=of a8]   {};


        \node[draw,rectangle,minimum width=40,minimum height=20,fill=solarizedred]    (b1)     [right=of a9]   {};
        \node[draw,rectangle,minimum width=40,minimum height=20,fill=solarizedred]    (b2)     [right=of b1]   {};

        \node[draw,rectangle,minimum width=40,minimum height=20]    (b3)     [right=of b2]   {};
        \node[draw,rectangle,minimum width=40,minimum height=20]    (b4)     [right=of b3]   {};

        \node[draw,rectangle,minimum width=90,minimum height=20,xshift=35]    (s0)    [below=of a0]   {Slab A1};
        \node[draw,rectangle,minimum width=90,minimum height=20]    (s1)    [right=of s0]   {Slab A2};
        \node[draw,rectangle,minimum width=80,minimum height=20]    (s2)    [right=of s1]   {Slab B1};
        \node[draw,rectangle,minimum width=80,minimum height=20]    (s3)    [right=of s2]   {Slab B2};

    \end{tikzpicture}

    \vspace{2em}

    We can reduce internal fragmentation if Slabs are located
    adjacently

    \hspace{2em} In this example A has internal fragmentation (dark box)
  \end{frame}

  \begin{frame}
    \frametitle{The Kernel Has To Implement It's Own Memory Allocations}

    The concepts are the same for user space memory allocation

    (the kernel just gives them more contiguous virtual memory pages):

    \begin{itemize}
      \item There's static and dynamic allocations
      \item For dynamic allocations, fragmentation is a big concern
      \item Dynamic allocation returns blocks of memory
        \begin{itemize}
          \item Fragmentation between blocks is external
          \item Fragmentation within a blocks is internal
        \end{itemize}
      \item There's 3 general allocation strategies for different sized
            allocations
        \begin{itemize}
          \item Best fit
          \item Worst fit
          \item First fit
        \end{itemize}
      \item Buddy allocator is a real world restricted implementation
      \item Slab allocator takes advantage of fixed sized objects to reduce
            fragmentation
    \end{itemize}
  \end{frame}
\end{document}
