\documentclass[aspectratio=169]{beamer}

\usepackage{ccicons}
\usepackage{fontspec}
\usepackage{import}
\usepackage{listings}
\usepackage{tikz}

\subimport{../}{colors.tex}

\usetikzlibrary{
  arrows,
  arrows.meta,
  automata,
  backgrounds,
  calc,
  chains,
  decorations.pathreplacing,
  fit,
  matrix,
  overlay-beamer-styles,
  positioning,
  shapes,
  tikzmark,
}
\usetikzmarklibrary{listings}

\hypersetup{
  colorlinks=true,
  urlcolor=uclablue,
}

\setbeamercolor{frametitle}{fg=primarycolor}
\setbeamercolor{structure}{fg=primarycolor}
\setbeamercolor{enumerate item}{fg=black}
\setbeamercolor{itemize item}{fg=black}
\setbeamercolor{itemize subitem}{fg=black}

\setbeamersize{text margin left=26.6mm}
\addtolength{\headsep}{2mm}

\setbeamertemplate{navigation symbols}{}
\setbeamertemplate{headline}{}
\setbeamertemplate{footline}{}
\setbeamertemplate{itemize item}{\color{black}}
\setbeamertemplate{itemize items}[circle]

\setbeamertemplate{footline}{
  \begin{tikzpicture}[remember picture,
                      overlay,
                      shift={(current page.south west)}]
    \node [black!50, inner sep=2mm, anchor=south east]
          at (current page.south east) {\footnotesize \insertframenumber};
  \end{tikzpicture}
}

\setsansfont{Overpass}[Scale=MatchLowercase]
\setmonofont{Overpass Mono}[Scale=MatchLowercase]

\makeatletter
\newcommand\version[1]{\renewcommand\@version{#1}}
\newcommand\@version{}
\def\insertversion{\@version}

\newcommand\lecturenumber[1]{\renewcommand\@lecturenumber{#1}}
\newcommand\@lecturenumber{}
\def\insertlecturenumber{\@lecturenumber}
\makeatother

\setbeamertemplate{title page}
{
  \begin{tikzpicture}[remember picture,
                      overlay,
                      shift={(current page.south west)},
                      background rectangle/.style={fill=uclablue},
                      show background rectangle]
    \node [anchor=west, align=left, inner sep=0, text=white]
          (lecturenumber) at (\paperwidth / 6, \paperheight * 3 / 4)
          {\Large Lecture \insertlecturenumber};
    \node [inner sep=0, align=left, text=white, node distance=0,
           above left=of lecturenumber, anchor=south west, yshift=2mm]
          {\Large CS 111: Operating System Principles};
    \node (title) [inner sep=0, anchor=west, align=right, text=white]
          at (\paperwidth / 6, \paperheight / 2)
          {{\bfseries \Huge \inserttitle{}}};
    \node [inner sep=0, align=right, text=white, node distance=0,
           below right=of title, anchor=north east, yshift=-1mm]
          {{\footnotesize \ttfamily \insertversion}};
    \node [inner sep=0, text=white, align=left, anchor=west]
          (author) at (\paperwidth / 6, \paperheight / 4)
          {\insertauthor};
    \node [text=white, inner sep=0, align=left, node distance=0,
           below left=of author, anchor=north west, yshift=-2mm]
          {\insertdate};
    \node [align=right, anchor=south east, inner sep=2mm, text=white]
          (license) at (\paperwidth, 0)
          {\footnotesize This  work is licensed under a
           \href{http://creativecommons.org/licenses/by-sa/4.0/}
                {\color{uclagold} Creative Commons Attribution-ShareAlike 4.0
                 International License}};
    \node [text=white, inner sep=0, align=right, node distance=0,
           above right=of license, anchor=south east, xshift=-2mm]
          {\Large \ccbysa};
  \end{tikzpicture}
}

\tikzset{
  >=Straight Barb[],
  shorten >=1pt,
  initial text=,
}

\lstset{
  basicstyle=\footnotesize\ttfamily,
}


\lecturenumber{22}
\title{Course Recap}
\version{1.0.0}
\author{Jon Eyolfson}
\date{June 1, 2021}

\begin{document}
  \begin{frame}[plain, noframenumbering]
    \titlepage
  \end{frame}

  \begin{frame}
    \frametitle{There are 4 Major Concepts in This Course}

    You'll learn how the following applies to operating systems:

    \begin{itemize}
      \item Virtualization
      \item Concurrency
      \item Persistence
      \item Security (out of scope, somewhat touched on in Virtual Machines)
    \end{itemize}
  \end{frame}

  \begin{frame}
    \frametitle{Kernel Interfaces Operate Between CPU Mode Boundaries}

    The lessons from the lecture:
    \begin{itemize}
      \item Code running in kernel mode is part of your kernel
      \item Different kernel architectures shift how much code runs in kernel mode
      \item System calls are the interface between user and kernel mode
      \item Everything involved to define a simple ``Hello world'' (in 178 bytes)
      \begin{itemize}
        \item Difference between API and ABI
        \item How to explore system calls
      \end{itemize}
    \end{itemize}
  \end{frame}

  \begin{frame}
    \frametitle{Operating Systems Provide the Foundation for Libraries}

    We learned:
    \begin{itemize}
      \item Dynamic libraries and a comparison to static libraries
      \begin{itemize}
        \item How to manipulate the dynamic loader
      \end{itemize}
      \item Example of issues from ABI changes without API changes
      \item Standard file descriptor conventions for UNIX
    \end{itemize}
  \end{frame}

  \begin{frame}
    \frametitle{The Operating System Creates and Runs Processes}

    The operating system has to:
    \begin{itemize}
      \item Loads a program, and create a process with context
      \item Maintain process control blocks, including state
      \item Switch between running processes using a context switch
      \item Unix kernels start an \texttt{init} process
      \item Unix processes have to maintain a parent and child relationship
    \end{itemize}
  \end{frame}

  \begin{frame}
    \frametitle{We Used System Calls to Create Processes}

    You should be comfortable with:
    \begin{itemize}
      \item \texttt{execve}
      \item \texttt{fork}
      \item \texttt{wait}
    \end{itemize}

    \vspace{2em}

    This includes understanding processes and their relationships
  \end{frame}

  \begin{frame}
    \frametitle{We Explored Basic IPC in an Operating System}

    Some basic IPC includes:
    \begin{itemize}
      \item \texttt{read} and \texttt{write} through file descriptors (could be a regular file)
      \item Redirecting file descriptors for communcation
      \item Pipes (which you'll explore)
      \item Signals
      \item Shared Memory
    \end{itemize}
  \end{frame}

  \begin{frame}
    \frametitle{Scheduling Involves Trade-Offs}

    We looked at few different algorithms:
    \begin{itemize}
      \item First Come First Served (FCFS) is the most basic scheduling algorithm
      \item Shortest Job First (SJF) is a tweak that reduces waiting time
      \item Shortest Remaining Time First (SRTF) uses SJF ideas with preemptions
      \item SRTF optimizes lowest waiting time (or turnaround time)
      \item Round-robin (RR) optimizes fairness and response time
    \end{itemize}
  \end{frame}

  \begin{frame}
    \frametitle{Scheduling Gets Even More Complex}

    There are more solutions, and more issues:
    \begin{itemize}
      \item Introducing priority also introduces priority inversion
      \item Some processes need good interactivity, others not so much
      \item Multiprocessors may require per-CPU queues
      \item Real-time requires predictability
      \item Completely Fair Scheduler (CFS) tries to model the ideal fairness
    \end{itemize}
  \end{frame}

  \begin{frame}
    \frametitle{Page Tables Translate Virtual to Physical Addresses}

    The MMU is the hardware that uses page tables, which may:
    \begin{itemize}
      \item Be a single large table (wasteful, even for 32-bit machines)
      \item Be a multi-level to save space for sparse allocations
      \item Use the kernel allocate pages from a free list
      \item Use a TLB to speed up memory accesses
    \end{itemize}
  \end{frame}

  \begin{frame}
    \frametitle{Page Replacement Algorithms Aim to Reduce Page Faults}

    We saw the following:
    \begin{itemize}
      \item Optimal (good for comparison but not realistic)
      \item Random (actually works surprisingly well, avoids the worst case)
      \item FIFO (easy to implement but Bélády's anomaly)
      \item LRU (gets close to optimal but expensive to implement)
      \item Clock (a decent approximation of LRU)
    \end{itemize}
  \end{frame}

  \begin{frame}
    \frametitle{Both Processes and (Kernel) Threads Enable Parallelization}

    We explored threads, and related them to something we already know (processes)
    \begin{itemize}
      \item Threads are lighter weight, and share memory by default
      \item Each process can have multiple (kernel) threads
      \item Most implementations use one-to-one user-to-kernel thread mapping
      \item The operating system has to manage what happens during a fork, or signals
      \item We now have synchronization issues
    \end{itemize}
  \end{frame}

  \begin{frame}
    \frametitle{We Want Critical Sections to Protect Against Data Races}

    We should know what data races are, and how to prevent them:
    \begin{itemize}
      \item Mutex or spinlocks are the most straightforward locks
      \item We need hardware support to implement locks
      \item We need some kernel support for wake up notifications
      \item If we know we have a lot of readers, we should use a read-write lock
    \end{itemize}
  \end{frame}

  \begin{frame}
    \frametitle{We Explored More Advanced Locking}

    Before we did mutual exclusion, now we can ensure order

    \begin{itemize}
      \item Semaphores are an atomic value that can be used for signaling
      \item Condition variables are clearer for complex condition signaling
      \item Locking granularity matters, you'll find out in Lab 3
      \item You must prevent deadlocks
    \end{itemize}
  \end{frame}

  \begin{frame}
    \frametitle{The Kernel Has To Implement It's Own Memory Allocations}

    The concepts are the same for user space memory allocation

    (the kernel just gives them more contiguous virtual memory pages):

    \begin{itemize}
      \item There's static and dynamic allocations
      \item For dynamic allocations, fragmentation is a big concern
      \item Dynamic allocation returns blocks of memory
        \begin{itemize}
          \item Fragmentation between blocks is external
          \item Fragmentation within a blocks is internal
        \end{itemize}
      \item There's 3 general allocation strategies for different sized
            allocations
        \begin{itemize}
          \item Best fit
          \item Worst fit
          \item First fit
        \end{itemize}
      \item Buddy allocator is a real world restricted implementation
      \item Slab allocator takes advantage of fixed sized objects to reduce
            fragmentation
    \end{itemize}
  \end{frame}

  \begin{frame}{Disks Enable Persistence}

    We explored two kinds of disks: SSDs and HDDs
    \begin{itemize}
      \item Magnetic disks have poor random access (need to be scheduled)
      \item Shortest Positioning Time First (SPTF) is the best scheduling for throughput
      \item SSDs are more like RAM except accessed in pages and blocks
      \item SSDs also need to work with the OS for best performance (TRIM)
      \item Use RAID to tolerate failures and improve performance using multiple disks
    \end{itemize}
  \end{frame}

  \begin{frame}{Filesystems Enable Persistence}
    They describe how files are stored on disks:
    \begin{itemize}
      \item API-wise you can open files, and change the position to read/write
            at
      \item Each process has a local open file and there's a global open file
            table
      \item There's multiple allocation strategies: contiguous, linked, FAT, indexed
      \item Linux uses a hybrid inode approach
      \item Everything is a file on UNIX, names in a directory can be hard or soft links
      
    \end{itemize}
  \end{frame}

  \begin{frame}{Sockets are IPC Across Physical Machines}

    We can now create servers and clients, but there's much more to learn!

       \hspace{2em} There's networking and distributed systems courses

    \vspace{2em}

    However, today we learned the basics:

    \begin{itemize}
      \item Sockets require an address (e.g. local and IPv4/IPv6)
      \item There are two types of sockets: stream and datagram
      \item Servers need to bind to an address, listen, and accept connections
      \item Clients need to connect to an address
    \end{itemize}
  \end{frame}

  \begin{frame}{Virtual Machines Virtualize a Physical Machine}

    They allow multiple operating systems to share the same hardware
    \begin{itemize}
      \item Virtual machines provide isolation, the hypervisor allocates
            resources
      \item Type 2 hypervisors are slower due to trap-and-emulate and binary
            translation
      \item Type 1 hypervisors are supported by hardware, IOMMU speeds up devices
      \item Hypervisors may overcommit resources and need to physically move VM
      \item Containers aim to have the benefits of VMs, without the overhead
    \end{itemize}
  \end{frame}

  \begin{frame}{Next Lecture We'll Go Over Example Final Questions}

    Take the extra time to wrap up Lab 4!

    \begin{itemize}
      \item Everything is valid to put on the final
      \item Same idea as the midterm, just 50\% larger
      \item June 8th, 24 hour window
      \item I'll post the 2020 final on Discord for you to go over
      \item Bring questions for the last lecture!
    \end{itemize}
  \end{frame}
\end{document}
