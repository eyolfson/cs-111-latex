\documentclass[aspectratio=169]{beamer}

\usepackage{ccicons}
\usepackage{fontspec}
\usepackage{import}
\usepackage{listings}
\usepackage{tikz}

\subimport{../}{colors.tex}

\usetikzlibrary{
  arrows,
  arrows.meta,
  automata,
  backgrounds,
  calc,
  chains,
  decorations.pathreplacing,
  fit,
  matrix,
  overlay-beamer-styles,
  positioning,
  shapes,
  tikzmark,
}
\usetikzmarklibrary{listings}

\hypersetup{
  colorlinks=true,
  urlcolor=uclablue,
}

\setbeamercolor{frametitle}{fg=primarycolor}
\setbeamercolor{structure}{fg=primarycolor}
\setbeamercolor{enumerate item}{fg=black}
\setbeamercolor{itemize item}{fg=black}
\setbeamercolor{itemize subitem}{fg=black}

\setbeamersize{text margin left=26.6mm}
\addtolength{\headsep}{2mm}

\setbeamertemplate{navigation symbols}{}
\setbeamertemplate{headline}{}
\setbeamertemplate{footline}{}
\setbeamertemplate{itemize item}{\color{black}}
\setbeamertemplate{itemize items}[circle]

\setbeamertemplate{footline}{
  \begin{tikzpicture}[remember picture,
                      overlay,
                      shift={(current page.south west)}]
    \node [black!50, inner sep=2mm, anchor=south east]
          at (current page.south east) {\footnotesize \insertframenumber};
  \end{tikzpicture}
}

\setsansfont{Overpass}[Scale=MatchLowercase]
\setmonofont{Overpass Mono}[Scale=MatchLowercase]

\makeatletter
\newcommand\version[1]{\renewcommand\@version{#1}}
\newcommand\@version{}
\def\insertversion{\@version}

\newcommand\lecturenumber[1]{\renewcommand\@lecturenumber{#1}}
\newcommand\@lecturenumber{}
\def\insertlecturenumber{\@lecturenumber}
\makeatother

\setbeamertemplate{title page}
{
  \begin{tikzpicture}[remember picture,
                      overlay,
                      shift={(current page.south west)},
                      background rectangle/.style={fill=uclablue},
                      show background rectangle]
    \node [anchor=west, align=left, inner sep=0, text=white]
          (lecturenumber) at (\paperwidth / 6, \paperheight * 3 / 4)
          {\Large Lecture \insertlecturenumber};
    \node [inner sep=0, align=left, text=white, node distance=0,
           above left=of lecturenumber, anchor=south west, yshift=2mm]
          {\Large CS 111: Operating System Principles};
    \node (title) [inner sep=0, anchor=west, align=right, text=white]
          at (\paperwidth / 6, \paperheight / 2)
          {{\bfseries \Huge \inserttitle{}}};
    \node [inner sep=0, align=right, text=white, node distance=0,
           below right=of title, anchor=north east, yshift=-1mm]
          {{\footnotesize \ttfamily \insertversion}};
    \node [inner sep=0, text=white, align=left, anchor=west]
          (author) at (\paperwidth / 6, \paperheight / 4)
          {\insertauthor};
    \node [text=white, inner sep=0, align=left, node distance=0,
           below left=of author, anchor=north west, yshift=-2mm]
          {\insertdate};
    \node [align=right, anchor=south east, inner sep=2mm, text=white]
          (license) at (\paperwidth, 0)
          {\footnotesize This  work is licensed under a
           \href{http://creativecommons.org/licenses/by-sa/4.0/}
                {\color{uclagold} Creative Commons Attribution-ShareAlike 4.0
                 International License}};
    \node [text=white, inner sep=0, align=right, node distance=0,
           above right=of license, anchor=south east, xshift=-2mm]
          {\Large \ccbysa};
  \end{tikzpicture}
}

\tikzset{
  >=Straight Barb[],
  shorten >=1pt,
  initial text=,
}

\lstset{
  basicstyle=\footnotesize\ttfamily,
}


\labnumber{1}
\title{Pipe Up}
\version{3.0.0}
\author{Jon Eyolfson}
\date{October 6, 2021}
\duedate{October 20, 2021 @ 11:59 PM PT}

\begin{document}

\maketitle

In this lab you'll be writing the low level code performed by the pipe
(\texttt{|}) operator in shells.
Users will pass in executable names as command line arguments and you will
execute each one in a new process (again similar to a shell).
You're expected to understand your own implementation and test it yourself.
Lecture 5 and 6 cover the function calls you should need (except as noted
below).

\paragraph{Additional APIs.}

Given you are just using the executable names, you may use the C library helper
functions for \texttt{execve}, such as \texttt{execlp}.
\texttt{execlp} will let you skip using string arrays (using C varargs), and it
will also search for executables using the \texttt{PATH} environment variable.
You may notice your program hanging waiting for input.
You must call \texttt{close} on any file descriptors not explictly used in
your process.
Failure to call \texttt{close} will inform the operating system you are not done
with it, and it will never return end-of-file from a \texttt{read} system call.

\paragraph{Starting the lab.}

You'll receieve all future labs through my repository (Lab 0 had you set it up
as \texttt{upstream}).
Run the following command to get the skeleton for Lab 1:
\texttt{git pull upstream main}.
You should be able to run \texttt{make} in the \texttt{lab1} directory to
create a \texttt{pipe} executable, and then \texttt{make clean} to remove all
binary files.

\paragraph{Files to modify.}

You should only be modifying \texttt{pipe.c} and \texttt{README.md} in the
\texttt{lab1} directory.

\paragraph{Your task.}

You should execute the programs in \texttt{argv[1]}, ...,
\texttt{argv[argc - 1]} as new processes.
You also need to create a pipe between two subsequent processes.
For example, a pipe should connect \texttt{argv[1]}'s standard output to
\texttt{argv[2]}'s standard input (if there are at least two procsses).
The standard input of first new process must be the same as the standard input
of the parent process (\texttt{pipe}).
Also, the standard output of your last new process must be the same as the
standard output of the parent process.
You should be able to handle between 1 to at least 8 programs (more is okay).
All standard errors should be the same as the parent's standard error.
You do not need to handle passing additional command line arguments to every
individual new process.
All your processes should be a direct child of the original process that starts
executing \texttt{main}.
Finally, fill in your \texttt{README.md} so that you could use your program
without having to use this document.

\paragraph{Errors.}

Your program should (of course) handle errors from all function calls you make.
Your program should exit with the proper \texttt{errno} of the failing call.
It is okay to exit as soon as you find an error, without any error recovery.
If there are no programs as command line arguments, your program should exit
with errno \texttt{EINVAL} (invalid argument).
Your program should work with a single program as a command line argument.
Your program should not create any orphan processes (you must \texttt{wait}).

\paragraph{Tips.}

You should come up with smaller subtasks yourself.
One approach may be to execute one program from the command line, then multiple
programs independently. Afterwards set up your pipe between two processes, then
multiple processes.
Start small then work big.

\paragraph{Example output.}
Your output should be the same as if you were to use \texttt{|} between each
program in your shell.

\begin{lstlisting}
> ./pipe ls
Makefile  pipe	pipe.c	pipe.o	README.md
> ./pipe ls cat wc
      5       5      38
\end{lstlisting}

The last command should have the same output of: \texttt{ls | cat | wc}.

\paragraph{Testing.}

There are a set of basic test cases given to you (they'll be released a few
days after the lab).
We'll withhold more advanced tests which we'll use for grading.
Part of programming is coming up with tests yourself.
To run the provided test cases please run the following command in your lab
directory:

\begin{lstlisting}[xleftmargin=2em]
python -m unittest
\end{lstlisting}

\paragraph{Submission.}

Simply push your code using \lstinline|git push origin main| (or simply
\lstinline|git push|) to submit it.
\textit{You need to create your own commits to push, you can use as many
as you'd like.}
You'll need to use the \texttt{git add} and \texttt{git commit} commands.
You may push as many commits as you want, your latest commit that modifies
the lab files counts as your submission.
For late days we will look at the timestamp on our server.
We will never use your commit times (or file access times) as proof of
submission, only when you push your code to the course Git server.

We've created a new system for the fall quarter that double checks you
have the latest upstream code and have submitted something.
Please check
\url{https://laforge.cs.ucla.edu/cs111/grades/}
to see your status.
You're expected to properly merge in upstream code without rebasing.
Note that the website only updates your lab modification status
if you've merged the latest code.

\end{document}
