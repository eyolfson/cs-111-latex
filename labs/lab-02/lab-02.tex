\documentclass[aspectratio=169]{beamer}

\usepackage{ccicons}
\usepackage{fontspec}
\usepackage{import}
\usepackage{listings}
\usepackage{tikz}

\subimport{../}{colors.tex}

\usetikzlibrary{
  arrows,
  arrows.meta,
  automata,
  backgrounds,
  calc,
  chains,
  decorations.pathreplacing,
  fit,
  matrix,
  overlay-beamer-styles,
  positioning,
  shapes,
  tikzmark,
}
\usetikzmarklibrary{listings}

\hypersetup{
  colorlinks=true,
  urlcolor=uclablue,
}

\setbeamercolor{frametitle}{fg=primarycolor}
\setbeamercolor{structure}{fg=primarycolor}
\setbeamercolor{enumerate item}{fg=black}
\setbeamercolor{itemize item}{fg=black}
\setbeamercolor{itemize subitem}{fg=black}

\setbeamersize{text margin left=26.6mm}
\addtolength{\headsep}{2mm}

\setbeamertemplate{navigation symbols}{}
\setbeamertemplate{headline}{}
\setbeamertemplate{footline}{}
\setbeamertemplate{itemize item}{\color{black}}
\setbeamertemplate{itemize items}[circle]

\setbeamertemplate{footline}{
  \begin{tikzpicture}[remember picture,
                      overlay,
                      shift={(current page.south west)}]
    \node [black!50, inner sep=2mm, anchor=south east]
          at (current page.south east) {\footnotesize \insertframenumber};
  \end{tikzpicture}
}

\setsansfont{Overpass}[Scale=MatchLowercase]
\setmonofont{Overpass Mono}[Scale=MatchLowercase]

\makeatletter
\newcommand\version[1]{\renewcommand\@version{#1}}
\newcommand\@version{}
\def\insertversion{\@version}

\newcommand\lecturenumber[1]{\renewcommand\@lecturenumber{#1}}
\newcommand\@lecturenumber{}
\def\insertlecturenumber{\@lecturenumber}
\makeatother

\setbeamertemplate{title page}
{
  \begin{tikzpicture}[remember picture,
                      overlay,
                      shift={(current page.south west)},
                      background rectangle/.style={fill=uclablue},
                      show background rectangle]
    \node [anchor=west, align=left, inner sep=0, text=white]
          (lecturenumber) at (\paperwidth / 6, \paperheight * 3 / 4)
          {\Large Lecture \insertlecturenumber};
    \node [inner sep=0, align=left, text=white, node distance=0,
           above left=of lecturenumber, anchor=south west, yshift=2mm]
          {\Large CS 111: Operating System Principles};
    \node (title) [inner sep=0, anchor=west, align=right, text=white]
          at (\paperwidth / 6, \paperheight / 2)
          {{\bfseries \Huge \inserttitle{}}};
    \node [inner sep=0, align=right, text=white, node distance=0,
           below right=of title, anchor=north east, yshift=-1mm]
          {{\footnotesize \ttfamily \insertversion}};
    \node [inner sep=0, text=white, align=left, anchor=west]
          (author) at (\paperwidth / 6, \paperheight / 4)
          {\insertauthor};
    \node [text=white, inner sep=0, align=left, node distance=0,
           below left=of author, anchor=north west, yshift=-2mm]
          {\insertdate};
    \node [align=right, anchor=south east, inner sep=2mm, text=white]
          (license) at (\paperwidth, 0)
          {\footnotesize This  work is licensed under a
           \href{http://creativecommons.org/licenses/by-sa/4.0/}
                {\color{uclagold} Creative Commons Attribution-ShareAlike 4.0
                 International License}};
    \node [text=white, inner sep=0, align=right, node distance=0,
           above right=of license, anchor=south east, xshift=-2mm]
          {\Large \ccbysa};
  \end{tikzpicture}
}

\tikzset{
  >=Straight Barb[],
  shorten >=1pt,
  initial text=,
}

\lstset{
  basicstyle=\footnotesize\ttfamily,
}


\labnumber{2}
\title{You Spin Me Round Robin}
\version{1.0.3}
\author{Jon Eyolfson}
\date{April 26, 2021}
\duedate{May 7, 2021 at 8 PM PST\\Effective due date: May 10, 2021 at 8 PM PST (due to a typo)}

\begin{document}

\maketitle

In this lab you'll be writing the implementation for round robin scheduling
for a given workload and quantum length.
You'll be given a basic skeleton that parses an input file and command line
arguments.
You're expected to understand how you would implement round robin if you were
to implement it yourself in a kernel (which means doing it in C).
Lecture 9 gave a quick introduction into how to use the C style linked lists
and the structure of the skeleton code.

\paragraph{Additional APIs.}

You may need a doubly linked list for your implement.
For this lab you should use \texttt{TAILQ} from \texttt{sys/queue.h}.
Use \texttt{man 3 tailq} to see all of the macros you can use.
There's already a list created for you called \texttt{process\_list} with a
\texttt{TAILQ} entry name of \texttt{pointers}.
You should not have to include any more headers or use any additional APIs,
besides adding your code.

\paragraph{Starting the lab.}

Run the following command to get the skeleton for Lab 2:
\texttt{git pull upstream main}.
You should be able to run \texttt{make} in the \texttt{lab-02} directory to
create a \texttt{rr} executable, and then \texttt{make clean} to remove all
binary files.
There is also an example \texttt{processes.txt} file in your lab directory.
The \texttt{rr} executable takes a file path as the first argument, and a
quantum length as the second.
For example, you can run: \texttt{./rr processes.txt 3}.

\paragraph{Files to modify.}

You should only be modifying \texttt{rr.c} and \texttt{README.md} in the
\texttt{lab-02} directory.

\paragraph{Your task.}

You should only add additional fields to \texttt{struct process} and add your
code to \texttt{main} between the comments in the skeleton. You may add
functions to call from main if you wish, but calls should only be between the
comments. We assume a single time unit is the smallest atomic unit (we cannot
split it even smaller).
You should ensure your scheduler calculates the total waiting time and total
response time to the variables \texttt{total\_waiting\_time} and
\texttt{total\_response\_time}.
The program then outputs the average waiting time and response time for you.
Finally, fill in your \texttt{README.md} so that you could use your program
without having to use this document.

\paragraph{Errors.}

All the allocations and input are handled for you, so there should be no need to
handle any errors.
You may assume integer overflows will not happen with a valid schedule.

\paragraph{Tips.}

You should ensure your implementation works with the examples in Lecture 7.

\paragraph{Example output.}

The \texttt{process.txt} file is the example from Lecture 7. You should be able
run:

\begin{lstlisting}
> ./rr processes.txt 3
Average waiting time: 7.00             
Average response time: 2.75
\end{lstlisting}

\paragraph{Submission.}

Simply push your code using \lstinline|git push origin main| (or simply
\lstinline|git push|).
For late days will we look at the timestamp on our server.
We will never use your commit times as proof of submission, only when you
push your code to the course Git server.

\end{document}
