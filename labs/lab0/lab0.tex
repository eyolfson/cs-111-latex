\documentclass{article}

\usepackage{enumitem}
\usepackage{fontspec}
\usepackage[letterpaper,margin=72pt]{geometry}
\usepackage{hyperref}
\usepackage{import}
\usepackage{listings}
\usepackage{xcolor}

\subimport{../}{colors.tex}

\setsansfont{Overpass}[Scale=MatchLowercase]
\setmonofont{Overpass Mono}[Scale=MatchLowercase]

\renewcommand{\familydefault}{\sfdefault}

\hypersetup{
  colorlinks=true,
  urlcolor=uclablue,
}

\setlist{nosep}

\makeatletter
\newcommand\version[1]{\renewcommand\@version{#1}}
\newcommand\@version{}

\newcommand\labnumber[1]{\renewcommand\@labnumber{#1}}
\newcommand\@labnumber{}

\newcommand\duedate[1]{\renewcommand\@duedate{#1}}
\newcommand\@duedate{}

\renewcommand\maketitle{%
  \noindent
  {\Large \color{uclablue} CS 111: Operating System Principles}

  \noindent
  {\Large \color{uclablue} Lab \@labnumber}\\[-0.75em]

  \noindent
  {\Huge \bfseries \color{uclablue} \@title}
  {\ttfamily \footnotesize \color{uclablue} \@version}\\[-0.75em]

  \noindent
  {\@author}

  \noindent
  {\@date}

  \noindent
  {Due: \@duedate}\\[1em]
}
\makeatother

\lstset{
  basicstyle=\ttfamily,
}


\labnumber{0}
\title{A Kernel Seedling}
\version{3.0.0}
\author{Jon Eyolfson}
\date{September 29, 2021}
\duedate{October 6, 2021 @ 11:59 PM PT}

\begin{document}

\maketitle

In this lab, you'll setup a virtual machine and write your (probably) first
kernel module.
We'll use VirtualBox as our hypervisor since it supports many different host
operating systems, and is friendly to learn.
You'll be using Git to submit your work and save your progress.
Finally, you'll write a kernel module that adds a file to \lstinline|/proc/|
to expose internal kernel information.

\paragraph{Virtual machine setup.}

After the setup you'll have a fully functioning Linux virtual machine.
You're free to edit your files with whatever you're comfortable with.
For example, you can install VSCode with: \texttt{sudo pacman -S code}.
You may replace \texttt{code} with \texttt{emacs} or \texttt{vim} as well.
You should only run your code on the virtual machine.

\begin{enumerate}
  \item Download and install VirtualBox 6.1.26:
        \url{https://www.virtualbox.org/wiki/Downloads}
  \item Download our virtual machine:
        \url{https://laforge.cs.ucla.edu/cs111/media/cs111/vm.ova}
  \item Import the virtual machine
    \begin{enumerate}
      \item \textit{File} $\rightarrow$ \textit{Import Appliance}
      \item Choose \lstinline|vm.ova| from your local file system
      \item \textit{Next} $\rightarrow$ \textit{Import}
    \end{enumerate}
  \item Select \textit{CS 111} from the left panel and click \textit{Start} at
        the top of the right panel
  \item Use \lstinline|cs111| for both the username and password
  \item (Optional) Go to \textit{View} $\rightarrow$ \textit{Virtual Screen 1}
        and resize to any resolution you'd like
\end{enumerate}

\paragraph{Git setup.}
Run all these commands in your home directory (or anywhere really) on your
virtual machine.
First, open a terminal by going to \textit{Activities} and selecting the
\textit{Terminal} icon on the left.
If you're unfamiliar with Git, please check out the
\href{https://git-scm.com/book/en/v2/}{Pro Git} book.
For any of the commands, run them in the terminal.

\begin{enumerate}
  \item Run: \lstinline|git config --global user.name "Your Full Name"|
  \item Run: \lstinline|git config --global user.email your@email.com|
  \item Run: \lstinline|ssh-keygen -o|
    \begin{enumerate}
      \item Press \textit{Enter} for the default location
      \item Press \textit{Enter} for no passphrase
      \item Press \textit{Enter} again to confirm
    \end{enumerate}
  \item Login to the course website
    \begin{enumerate}
      \item (Optional) Go to \textit{Activities} and click the \textit{Firefox}
            icon on the left
    \end{enumerate}
  \item Click your username in the top right
  \item Click \textit{New SSH Key (Text Input)}
  \item Add your SSH key
    \begin{enumerate}
      \item Run: \lstinline|cat ~/.ssh/id_rsa.pub|
      \item Copy and paste the contents into the text box
      \item (Optional) Give the key a comment (it'll be its name)
      \item Press submit
    \end{enumerate}
  
  \item Run: \lstinline|cd ~|
  \item Run: \lstinline|git clone git@laforge.cs.ucla.edu:fall21/USERNAME/cs111|

        (replace \texttt{USERNAME} with your username)
  \item Run: \lstinline|cd cs111|
  \item Run: \lstinline|git remote add upstream git@laforge.cs.ucla.edu:fall21/jon/cs111|
\end{enumerate}

\paragraph{Lab Setup.}
Ensure you're in the repository (\lstinline|cd ~/cs111|) directory.
Make sure you have the latest skeleton code from us by running:
\lstinline|git pull upstream main|.
You can finally run: \lstinline|cd lab0| to begin the lab.

\paragraph{Your task.}

You're going to create a \lstinline|/proc/count| file that shows the current
number of running processes (or tasks) running.
The process table runs within kernel mode, so to access it you'll need to write
a kernel module that runs in kernel mode.
For your submission you'll modify \lstinline|proc_count.c|, and only this file,
for the coding part.
In the \lstinline|lab0| directory we should be able to run the following
commands:

\begin{lstlisting}[xleftmargin=2em]
make
sudo insmod proc_count.ko
cat /proc/count 
\end{lstlisting}

\noindent
The last command should report a single integer representing the number of
processes (or tasks) running on the machine.
Your final task is to fill in your documentation in the \lstinline|README.md|
for \lstinline|lab0|.

\paragraph{Tips.}

The kernel code is well commented, you can use \url{https://elixir.bootlin.com/}
for looking up functions and macros (symbols). There's already a skeleton that
uses: \lstinline|MODULE_AUTHOR|, \lstinline|MODULE_DESCRIPTION|,
\lstinline|MODULE_LICENSE|, \lstinline|module_init|, \lstinline|module_exit|,
and \lstinline|pr_info|.
You'll probably want to use the following to complete this lab:

\begin{lstlisting}[xleftmargin=2em]
proc_create_single
proc_remove
for_each_process
seq_printf
\end{lstlisting}

\noindent
You can divide this task into small subtasks:

\begin{enumerate}
  \item Properly create and remove \lstinline|/proc/count| when your module
        loads and unloads, respectively
  \item Make \lstinline|/proc/count| return some string when you
        \lstinline|cat /proc/count|
  \item Make \lstinline|/proc/count| return a integer with the number of running
        processes (or tasks) when you \lstinline|cat /proc/count|
\end{enumerate}

\paragraph{Commands.}

You'll have to use the following commands for this lab:

Build your module with \lstinline|make|

Insert your module into the kernel with \lstinline|sudo insmod proc_count.ko|

Read any information messages printed in the kernel with
\lstinline|sudo dmesg -l info|

Remove your module from the kernel (so you can insert a new one) with
\lstinline|sudo rmmod proc_count|

Sanity check your module information with \lstinline|modinfo proc_count.ko|

\paragraph{Testing.}

There are a set of basic test cases given to you.
For this lab the provided test cases are likely the ones we'll use for grading.
In the future we'll withhold more advancded tests which we'll use for grading.
Part of programming is coming up with tests yourself.
To run the provided test cases please run the following command in your lab
directory:

\begin{lstlisting}[xleftmargin=2em]
python -m unittest
\end{lstlisting}

\paragraph{Grading.}

The breakdown is as follows:

75\% \hspace{0.5em} code implementation in \lstinline|proc_count.c|

25\% \hspace{0.5em} documentation in \lstinline|README.md|

\paragraph{Submission.}

Simply push your code using \lstinline|git push origin main| (or simply
\lstinline|git push|) to submit it.
\textit{You need to create your own commits to push, you can use as many
as you'd like.}
You'll need to use the \texttt{git add} and \texttt{git commit} commands.
You may push as many commits as you want, your latest commit that modifies
the lab files counts as your submission.
For late days we will look at the timestamp on our server.
We will never use your commit times (or file access times) as proof of
submission, only when you push your code to the course Git server.

We've created a new system for the fall quarter that double checks you
have the latest upstream code and have submitted something.
Please check
\url{https://laforge.cs.ucla.edu/cs111/grades/}
to see your status.
You're expected to properly merge in upstream code without rebasing.
Note that the website only updates your lab modification status
if you've merged the latest code.

\end{document}
