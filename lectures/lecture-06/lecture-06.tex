\documentclass{article}

\usepackage{enumitem}
\usepackage{fontspec}
\usepackage[letterpaper,margin=72pt]{geometry}
\usepackage{hyperref}
\usepackage{import}
\usepackage{listings}
\usepackage{xcolor}

\subimport{../}{colors.tex}

\setsansfont{Overpass}[Scale=MatchLowercase]
\setmonofont{Overpass Mono}[Scale=MatchLowercase]

\renewcommand{\familydefault}{\sfdefault}

\hypersetup{
  colorlinks=true,
  urlcolor=uclablue,
}

\setlist{nosep}

\makeatletter
\newcommand\version[1]{\renewcommand\@version{#1}}
\newcommand\@version{}

\newcommand\labnumber[1]{\renewcommand\@labnumber{#1}}
\newcommand\@labnumber{}

\newcommand\duedate[1]{\renewcommand\@duedate{#1}}
\newcommand\@duedate{}

\renewcommand\maketitle{%
  \noindent
  {\Large \color{uclablue} CS 111: Operating System Principles}

  \noindent
  {\Large \color{uclablue} Lab \@labnumber}\\[-0.75em]

  \noindent
  {\Huge \bfseries \color{uclablue} \@title}
  {\ttfamily \footnotesize \color{uclablue} \@version}\\[-0.75em]

  \noindent
  {\@author}

  \noindent
  {\@date}

  \noindent
  {Due: \@duedate}\\[1em]
}
\makeatother

\lstset{
  basicstyle=\ttfamily,
}


\lecturenumber{6}
\title{Basic IPC}
\version{1.0.0}
\author{Jon Eyolfson}
\date{April 8, 2021}

\begin{document}
  \begin{frame}[plain, noframenumbering]
    \titlepage
  \end{frame}

  \begin{frame}
    \frametitle{IPC is Transferring Bytes Between Two or More Processes}

    Reading and writing files is a form of IPC

    \vspace{2em}

    For a process you can read the input, and write the output

    \hspace{2em} Think about Lab 0

    \vspace{2em}

    The read and write system calls allow any bytes
  \end{frame}

  \begin{frame}
    \frametitle{A Simple Process Could Write Everything It Reads}

    See: \texttt{lecture-06/read-write-example.c}

    \vspace{2em}

    We \texttt{read} from standard in, and \texttt{write} to standard out

    \hspace{2em} Does this remind you of any program you've seen before?

    \vspace{2em}

    If we run it in our terminal without arguments, it'll wait for input

    \hspace{2em} Press Ctrl+D when you're done to send end-of-file (EOF)
  \end{frame}

  \begin{frame}
    \frametitle{\texttt{read} Just Reads Data from a File Descriptor}

    See: \texttt{man 2 read}

    \vspace{2em}

    There's no EOF character, \texttt{read} just returns 0 bytes read

    \hspace{2em} The kernel returns 0 on a closed file descriptor

    \vspace{2em}

    We need to check for errors!

    \hspace{2em} Save \texttt{errno} if you're using another function that may
                 set it
  \end{frame}

  \begin{frame}
    \frametitle{\texttt{write} Just Writes Data to a File Descriptor}

    See: \texttt{man 2 write}

    \vspace{2em}

    It returns the number of bytes written, you can't assume it's always
    successful

    \hspace{2em} Save \texttt{errno} if you're using another function that may
                 set it

    \vspace{2em}

    Both ends of the read and write have a corresponding write and read

    \hspace{2em} This makes two communication channels with command line programs
  \end{frame}

  \begin{frame}
    \frametitle{The Standard File Descriptors Are Powerful}

    We could close standard input (freeing file descriptor 0) and open a file
    instead

    \hspace{2em} Linux uses the lowest available file descriptor for new ones

    \vspace{2em}

    See: \texttt{lecture-06/open-example.c} and \texttt{man 2 open}
    
    \vspace{2em}

    Without changing the core code, it now works with multiple input types

    \vspace{2em} You could type, or use a file
  \end{frame}

  \begin{frame}
    \frametitle{Your Shell Will Let You Redirect Standard File Descriptors}

    Instead of running \texttt{./open-example open-example.c} we could run:

    \hspace{2em} \texttt{./open-example < open-example.c}

    \vspace{2em}

    Your shell will do the \texttt{open} for you and replace the standard input

    \hspace{2em} We didn't actually have to write that!

    \vspace{2em}

    You could also redirect across multiple processes

    \hspace{2em} \texttt{cat open-example.c | ./open-example} 
  \end{frame}

  \begin{frame}
    \frametitle{Piping in Your Shell Connects Two Processes Together}

    In \texttt{./p1 | ./p2} the shell connects:
    \texttt{p1}'s \texttt{stdout} to \texttt{p2}'s \texttt{stdin}

    \vspace{2em}

    The kernel has a \texttt{pipe} system call that returns two file descriptors

    \hspace{2em} \texttt{fd\_pair[0]} is for \texttt{read} and
    \texttt{fd\_pair[1]} is for \texttt{write}

    \vspace{2em}

    This forms a one-way communication channel
  \end{frame}

  \begin{frame}
    \frametitle{Your Shell Properly Handles All the File Descriptors}

    This includes changing file descriptors, and closing them properly

    \vspace{2em}

    You can use \texttt{dup2} to move a file descriptor to a new one

    \hspace{2em} If the new one is already open, it'll \texttt{close} it first

    \vspace{2em}
    
    See: \texttt{man 2 dup2} and \texttt{man 2 close}

    \vspace{2em}

    How do you give processes different file descriptors?

    \hspace{2em} \texttt{fork} copies all the file descriptors to the new process
  \end{frame}

  \begin{frame}
    \frametitle{Signals are a Form of IPC that Interrupts}

    You could also press Ctrl+C to stop \texttt{./open-example}

    \hspace{2em} This interrupts your programs execution and exits early

    \vspace{2em}

    Kernel sends a number to your program indicating the type of signal
    
    \hspace{2em} Kernel default handlers either ignore the signal or terminate
    your process

    \vspace{2em}

    Ctrl+C sends \texttt{SIGINT} (interrupt from keyboard)

    \vspace{2em}

    If the default handler occurs the exit code will be 128 + signal number
  \end{frame}

  \begin{frame}
    \frametitle{You Can Set Your Own Signal Handlers with \texttt{signal}}

    See: \texttt{lecture-06/signal-example.c} and \texttt{man 2 signal}

    \vspace{2em}

    You just declare a function that doesn't return a value, and has an \texttt{int} argument

    \hspace{2em} The integer is the signal number

    \vspace{2em}

    Some numbers are non standard, here a few from Linux x86-64:
    \begin{itemize}
      \item 2: \texttt{SIGINT} (interrupt from keyboard)
      \item 9: \texttt{SIGKILL} (terminate immediately)
      \item 11: \texttt{SIGSEGV} (memory access violation)
      \item 15: \texttt{SIGTERM} (terminate)
    \end{itemize}
  \end{frame}

  \begin{frame}
    \frametitle{A Signal Pauses Your Process and Runs the Signal Handler}

    Your process can be interrupted at any point in execution

    \hspace{2em} Your process resumes after the signal handler finishes

    \vspace{2em}

    This is an example of concurrency, your process switches execution

    \hspace{2em} You have to be careful what you write here

    \vspace{2em}

    Run \texttt{./signal-example} and press Ctrl+C
  \end{frame}

  \begin{frame}
    \frametitle{You Need to Account for Interrupted System Calls}

    You should see:

    \hspace{2em} \texttt{Ignoring signal 2}

    \hspace{2em} \texttt{read: Interrupted system call}

    \vspace{2em}

    We can rewrite it to retry interrupted system calls

    \hspace{2em} See: \texttt{lecture-06/signal-example-2.c}

    \vspace{2em}

    Now the program continues when we press Ctrl+C
  \end{frame}

  \begin{frame}
    \frametitle{You Can Send Signals to Processes with Their PID}

    You can use the command: \texttt{kill}

    \hspace{2em} It is also a system call, taking a \texttt{pid} and signal number

    \vspace{2em}

    Find a processes' ID with \texttt{pidof}, e.g. \texttt{pidof ./signal-example-2}

    \vspace{2em}

    After use \texttt{kill <pid>}, which by default sends \texttt{SIGTERM}

    \vspace{2em}

    Use \texttt{kill -9 <pid>} to tell the kernel to terminate the process

    \hspace{2em} Process won't terminate if it's in uninterruptible sleep
  \end{frame}

  \begin{frame}
    \frametitle{Shared Memory Allows Two Processes to Access the Same Memory}

    See: \texttt{lecture-06/shared-memory-example.c} and \texttt{man 3 shm\_open}

    \vspace{2em}

    You use \texttt{shm\_open} which returns a file descriptor

    \vspace{2em}

    You can think of it as a new location to read and write bytes to

    \hspace{2em} This needs to be resized with \texttt{ftruncate}

    \vspace{2em}

    Note: on some implementations this just opens a file in \texttt{/dev/shm}
  \end{frame}

  \begin{frame}
    \frametitle{\texttt{mmap} Allows You to Memory Map the Contents of a File Descriptor}

    See: \texttt{man 3 mmap}

    \vspace{2em}

    Instead of using \texttt{read} and \texttt{write} system calls, you just access memory

    \hspace{2em} The operating system is responsible for management

    \vspace{2em}

    Instead of accessing the file sequentially, you can access any part of it

    \vspace{2em}

    You can \texttt{mmap} regular files as well!
  \end{frame}

  \begin{frame}
    \frametitle{We Explored Basic IPC in an Operating System}

    Some basic IPC includes:
    \begin{itemize}
      \item \texttt{read} and \texttt{write} through file descriptors (could be a regular file)
      \item Redirecting file descriptors for communcation
      \item Pipes (which you'll explore)
      \item Signals
      \item Shared Memory
    \end{itemize}
  \end{frame}
\end{document}
