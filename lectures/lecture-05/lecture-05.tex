\documentclass{article}

\usepackage{enumitem}
\usepackage{fontspec}
\usepackage[letterpaper,margin=72pt]{geometry}
\usepackage{hyperref}
\usepackage{import}
\usepackage{listings}
\usepackage{xcolor}

\subimport{../}{colors.tex}

\setsansfont{Overpass}[Scale=MatchLowercase]
\setmonofont{Overpass Mono}[Scale=MatchLowercase]

\renewcommand{\familydefault}{\sfdefault}

\hypersetup{
  colorlinks=true,
  urlcolor=uclablue,
}

\setlist{nosep}

\makeatletter
\newcommand\version[1]{\renewcommand\@version{#1}}
\newcommand\@version{}

\newcommand\labnumber[1]{\renewcommand\@labnumber{#1}}
\newcommand\@labnumber{}

\newcommand\duedate[1]{\renewcommand\@duedate{#1}}
\newcommand\@duedate{}

\renewcommand\maketitle{%
  \noindent
  {\Large \color{uclablue} CS 111: Operating System Principles}

  \noindent
  {\Large \color{uclablue} Lab \@labnumber}\\[-0.75em]

  \noindent
  {\Huge \bfseries \color{uclablue} \@title}
  {\ttfamily \footnotesize \color{uclablue} \@version}\\[-0.75em]

  \noindent
  {\@author}

  \noindent
  {\@date}

  \noindent
  {Due: \@duedate}\\[1em]
}
\makeatother

\lstset{
  basicstyle=\ttfamily,
}


\lecturenumber{5}
\title{Process API}
\version{3.0.0}
\author{Jon Eyolfson}
\date{Octoboer 7, 2021}

\begin{document}
  \begin{frame}[plain, noframenumbering]
    \titlepage
  \end{frame}

  \begin{frame}
    \frametitle{Linux Terminology Is Slightly Different}

    You can look at a process' state by reading \texttt{/proc/<pid>/state}

    \hspace{2em} Replace \texttt{<pid>} with the process ID

    \vspace{2em}

    R: Running and runnable [Running and Waiting]

    S: Interruptible sleep [Blocked]

    D: Uninterruptible sleep [Blocked]

    T: Stopped

    Z: Zombie

    \vspace{2em}

    The kernel lets you explicitly stop a process to prevent it from running

    \hspace{2em} You or another process must explicitly continue it
  \end{frame}

  \begin{frame}
    \frametitle{On POSIX Systems, You Can Find Documentation Using \texttt{man}}

    We'll be using the following APIs:
    \begin{itemize}
      \item \texttt{execve}
      \item \texttt{fork}
      \item \texttt{wait}
    \end{itemize}

    \vspace{2em}

    You can use \texttt{man <function>} to look up documentation,

    or \texttt{man <number> <function>}

    \hspace{2em} 2: System calls

    \hspace{2em} 3: Library calls
  \end{frame}

  \begin{frame}
    \frametitle{\texttt{execve} Loads Another Program, and Replaces Process with A New One}

    \texttt{execve} has the following API:
    \begin{itemize}
      \item \texttt{pathname}: Full path of the program to load
      \item \texttt{argv}: Array of strings (array of characters), terminated by a null pointer

            \hspace{2em} Represents arguments to the process
      \item \texttt{envp}: Same as \texttt{argv}

            \hspace{2em} Represents the environment of the process
      \item Returns an error on failure, does not return if successful
    \end{itemize}
  \end{frame}

  \begin{frame}[fragile]
    \frametitle{\texttt{execve-example.c} Turns the Process into \texttt{ls}}

    \begin{lstlisting}
int main(int argc, char *argv[]) {
  printf("I'm going to become another process\n");
  char *exec_argv[] = {"ls", NULL};
  char *exec_envp[] = {NULL};
  int exec_return = execve("/usr/bin/ls", exec_argv, exec_envp);
  if (exec_return == -1) {
    exec_return = errno;
    perror("execve failed");
    return exec_return;
  }
  printf("If execve worked, this will never print\n");
  return 0;
}
    \end{lstlisting}
  \end{frame}

  \begin{frame}
    \frametitle{\texttt{fork} Creates a New Process, A Copy of the Current One}

    \texttt{fork} as the following API:
    \begin{itemize}
      \item Returns the process ID of the newly created child process

            \hspace{2em} -1: on failure

            \hspace{2em} 0: in the child process

            \hspace{2em} >0: in the parent process
    \end{itemize}

    \vspace{2em}

    There are now 2 processes running

    \hspace{2em} Note: they can access the same variables, but they're separate

    \hspace{4em} Operating system does ``copy on write'' to maximize sharing
  \end{frame}

  \begin{frame}[fragile]
    \frametitle{\texttt{fork-example.c} Has One Process Execute Each Branch}

    \begin{lstlisting}
int main(int argc, char *argv[]) {
  pid_t pid = fork();
  if (pid == -1) {
    int err = errno;
    perror("fork failed");
    return err;
  }
  if (pid == 0) {
    printf("Child returned pid: %d\n", pid);
    printf("Child pid: %d\n", getpid());
    printf("Child parent pid: %d\n", getppid());
  }
  else {
    printf("Parent returned pid: %d\n", pid);
    printf("Parent pid: %d\n", getpid());
    printf("Parent parent pid: %d\n", getppid());
  }
  return 0;
}
    \end{lstlisting}
  \end{frame}

  \begin{frame}[fragile]
    \frametitle{\texttt{orphan-example.c} The Parent Exits Before the Child, \texttt{init} Cleans Up}

    \begin{lstlisting}
int main(int argc, char *argv[]) {
  pid_t pid = fork();
  if (pid == -1) {
    int err = errno;
    perror("fork failed");
    return err;
  }
  if (pid == 0) {
    printf("Child parent pid: %d\n", getppid());
    sleep(2);
    printf("Child parent pid (after sleep): %d\n", getppid());
  }
  else {
    sleep(1);
  }
  return 0;
}

    \end{lstlisting}
  \end{frame}

  \begin{frame}[fragile]
    \frametitle{\texttt{zombie-example.c} The Parent Monitors the Child To Check Its State}

    \begin{lstlisting}
  pid_t pid = fork();
  // Error checking
  if (pid == 0) {
    sleep(2);
  }
  else {
    // Parent process
    int ret;
    sleep(1);
    printf("Child process state: ");
    ret = print_state(pid);
    if (ret < 0) { return errno; }
    sleep(2);
    printf("Child process state: ");
    ret = print_state(pid);
    if (ret < 0) { return errno; }
  }
    \end{lstlisting}
  \end{frame}

  \begin{frame}
    \frametitle{You Need to Call \texttt{wait} on Child Processes}

    \texttt{wait} as the following API:
    \begin{itemize}
      \item \texttt{status}: Address to store the wait status of the process
      \item Returns the process ID of child process

            \hspace{2em} -1: on failure

            \hspace{2em} 0: for no blocking calls with no child changes

            \hspace{2em} >0: the child with a change
    \end{itemize}

    \vspace{2em}

    The wait status contains a bunch of information, including the exit code

    \hspace{2em} Use \texttt{man wait} to find all the macros to query wait status

    \hspace{4em} You can use \texttt{waitpid} to wait on a specific child process
  \end{frame}

  \begin{frame}[fragile]
    \frametitle{\texttt{wait-example.c} Blocks Until The Child Process Exists, and Cleans Up}

    \begin{lstlisting}
int main(int argc, char *argv[]) {
  pid_t pid = fork();
  if (pid == -1) {
    return errno;
  }
  if (pid == 0) {
    sleep(2);
  }
  else {
    printf("Calling wait\n");
    int wstatus;
    pid_t wait_pid = wait(&wstatus);
    if (WIFEXITED(wstatus)) {
      printf("Wait returned for an exited process! pid: %d, status: %d\n",
             wait_pid, WEXITSTATUS(wstatus));
    }
  }
  return 0;
}
    \end{lstlisting}
  \end{frame}

  \begin{frame}
    \frametitle{We Used System Calls to Create Processes}

    You should be comfortable with:
    \begin{itemize}
      \item \texttt{execve}
      \item \texttt{fork}
      \item \texttt{wait}
    \end{itemize}

    \vspace{2em}

    This includes understanding processes and their relationships
  \end{frame}
\end{document}
